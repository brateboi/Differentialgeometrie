\documentclass[../main.tex]{subfiles}

\begin{document}

\chapter{Die Geometrie der Gaussabbildung}

\section{Gaussabbildung}
\begin{definition}
    Sei $\Sigma \subset \mathbb{R}^{3}$ eine reguläre Fläche und $V \subset \Sigma$ offen. Eine stetige Abbildung $N:V \to S^3$ heisst \emph{Einheitsnormalenfeld} (oder lokale Gauss-Abbildung), falls 
    \begin{align*}
        \forall p \in V \text{ gilt: } N(q)\perp T_{p}\Sigma
    \end{align*}
\end{definition}
\begin{zusatz}
    Flächenelement $\sqrt{EG-F^{2}}$
    \begin{align*}
        EG-F^{2} = \det\begin{pmatrix} E & F \\ F & G \end{pmatrix}
    \end{align*}
\end{zusatz}
\begin{existence}
    Definiere $\begin{aligned}[t]
        N : V & \to S^{2} \\
        q & \mapsto \frac{\varphi{u}(q) \times \varphi_{v}(q)}{\lVert \varphi_{u}(q) \times \varphi_{v}(q) \rVert_{2}} 
    \end{aligned}$ \qquad \\
    Dieser Ausdruck ist stetig in $q$, da $\varphi_{u}$ und $\varphi_{v}$ stetig sind.
\end{existence}
\begin{uniqueness}
    Falls $V$ zusammenhängend ist, dan ist $N:V \to S^{2}$ bis auf Vorzeichen eindeutig festgelegt $:\pm N$.
\end{uniqueness}
\begin{remark}
    Falls $\Sigma \subset \mathbb{R}^{3}$ geschlossen ist (d.h. kompakt und ohne Rand), dann existiert sogar ein globales Einheitsnormalenfeld $N:\Sigma \to S^{2}$, genannt \emph{Gaussabbildung}.
    Tatsächlich trennt eine solche Fläche $\mathbb{R}^{3}$ in zwei Zusammenhangskomponenten. Für (nicht-geschlossene) reguläre Fläche $\Sigma \subset \mathbb{R}^{3}$ gilt: Es existiert $N:\Sigma \to S^{2}$ stetiges Einheitsnormalenfeld genau dann,
    wenn $\Sigma$ \emph{orientierbar} ist.
\end{remark}
\ctikzfig{zwei zeichnungen, torus und moebiusband}
Sei nun $\varphi : U \to V \subset \Sigma$ eine lokale $C^{1}$-Parametrisierung und $N:V \to S^{2}$ eine lokale Gaussabbildung. Dann gilt $\forall q \in V$:
\begin{itemize}
    \item $N(q)\perp T_{q}\Sigma$
    \item $N(q)\perp T_{N(q)}\Sigma$
\end{itemize}
$\implies T_{q}\Sigma = T_{N(q)}S^{2}$
für letzteres gilt $\forall p \in S^{2}: p\perp T_{p}S$\\ 
Falls $N:V \to S^{2}$ sogar differenzierbar ist, dann erhalten wir $\forall p \in V$ eine Abbildung
\begin{align*}
    (DN)_{p}: T_{p}\Sigma \to T_{N(p)}S^{2} = T_{p}\Sigma    
\end{align*}
die \emph{Weingartenabbildung}.

\begin{definition}
    \begin{align*}
        K(p) = \det{(DN)_{p}}\in \mathbb{R}\\
        \text{\emph{Gaussische Krümmung} im Punkt } p \in \Sigma
    \end{align*}
\end{definition}
\begin{remark}
    $K(p)$ hängt nicht von der Wahl von N ab, \\da $\det{-(DN)_{p}} = (-1)^{2} \cdot \det{(DN)_{p}}$ ist.
\end{remark}
\begin{example}
    \begin{enumerate}
        \item $\Sigma = \mathbb{R}^{2} \times \{0\} \subset \mathbb{R}^{3}\\
        \begin{aligned}
            N : \Sigma & \to S^{2}\\
            q & \mapsto e_{3} \text{(oder $-e_{3}$)}\\
        \end{aligned}\\
        N$ ist konstant, also gilt $\forall q \in \Sigma$ 
        $(DN)_{q} = 0; K(q) = 0$.
        \ctikzfig{K equiv to 1}
        \item $\Sigma = S^{2} \subset \mathbb{R}^{3}$ (Einheitssphäre)\\
        $\begin{aligned}
            N: S^{2} &\to S^{2}\\
            q &\mapsto q\\
        \end{aligned}$\\
        \ctikzfig{Einheitssphäre mit Krümmung = 1}
        $N = Id_{S^{2}}$ (oder $-Id_{S^{2}}$)\\
        $\forall q \in S^{2}$ gilt also $(DN)_{q} = Id : T_{q}\Sigma \to T_{q}\Sigma\\
        \implies K(q) = \det{Id: T_{q}\Sigma \to T_{q}\Sigma} = 1$
        \item $Z = {(x,y,z)\in \mathbb{R}^{3} | x^{2} + y^{2} = 1}\in \mathbb{R^{3}}\\
        \begin{aligned}
            N : Z &\to S^{2}\\
            (x,y,z) &\mapsto (x,y,0)\\
        \end{aligned}$\\
        Wir bemerken: $N$ hängt nicht von $z$ ab. 
        \ctikzfig{Zylinder mit K equiv to 0}
        Also gilt für alle $q \in Z : (DN)_{q}(e_{3}) = \lim\limits_{t \rightarrow 0}{\frac{\overbrace{N(q + t \cdot e_{3}) - N(q)}^{=0}}{t}} = 0$\\
        $\implies 0$ ist ein Eigenwert der Abbildung $(DN)_{q} : T_{q}Z \to T_{q}Z \implies K(q) = 0$.
    \end{enumerate}
    \begin{zusatz}
        Genauere Betrachtung des dritten Beispiels:\\
        Für $q = (x,y,z) \in Z$ gilt: $T_{q}Z = \text{span}\{e_{3}, \overbrace{-y \cdot e_{1} + x \cdot e_{2}}^{v}\}$\\
        Wir bestimmen $(DN)_{q}\stackrel{(\text{*})}{=} v$\\
        (*) Erklärung: Die Einschränkung von $N$ auf $S'\times \{0\}$ ist die Identität. Folglich ist die Abbildungsmatrix von $(DN)_{q}$ bezüglich der Basis $\{ e_{3}, -y \cdot e_{1} + x \cdot e_{2} \}$
    \end{zusatz}
\end{example}
\begin{definition}
    Die \emph{mittlere Krümmung} im Punkt $p \in \Sigma$ ist $H(p) = \frac{1}{2} \text{Spur}((DN)_{p})\in \mathbb{R}$, welche allerdings nur bis auf Vorzeichen definiert ist:
    \begin{align*}
        \text{Spur}(-(DN)_{p}) = - \text{Spur}(DN)_{p}
    \end{align*}

\end{definition}
\begin{remark}
    Reguläre Flächen mit $H \equiv 0$ heissen \emph{Minimalflächen}.
\end{remark}
\begin{examples}
    \begin{enumerate}
       \item $\Sigma = \mathbb{R}^{2} \times \{0\}. K\equiv 0 \text{ und } H\equiv 0$.
       \item $\Sigma = S^{2}$. $K \equiv 1 \text{ und } H\equiv 1 \impliedby \frac{1}{2}\text{ Spur}\begin{pmatrix}
           1 & 0 \\
           0 & 1
       \end{pmatrix}$
       \item $\Sigma = Z$. $K\equiv 0$ und $H\equiv \frac{1}{2} \impliedby \frac{1}{2} \cdot (\underbrace{\lambda_{1}}_{EW_{1}} + \underbrace{\lambda_{2}}_{EW_{2}}) = \frac{1}{2} \cdot (1 + 0)$ 
    \end{enumerate}
\end{examples}
\begin{notation}
    Ein Punkt $p\in \Sigma$ heisst:
    \begin{itemize}
        \item \emph{elliptisch}, falls $K(p) > 0$
        \item \emph{hyperbolisch}. falls $K(p) < 0$ (Sattelpunkt, siehe später)
        \item \emph{parabolisch}, falls $K(p) = 0 \text{ und } H(p) \neq 0$
        \item \emph{Flachpunkt}, falls $K(p) = 0 \text{ und } H(p) = 0$
    \end{itemize}
    \ctikzfig{minibeispiele zu all diesen}
\end{notation}
\begin{proposition}
    Sei $\Sigma \in \mathbb{R}^{3}$ eine reguläre Fläche, welche lokale $C^{2}$-Parametrisierungen besitzt (das heisst zweimal stetig differenzierbar). 
    Dann ist $\forall p \in \Sigma$ gilt:
    \begin{align*}
        \langle (DN)_{p}(a), b\rangle_{p} = \langle a, (DN)_{p}(b)\rangle_{p}
    \end{align*}
\end{proposition}
\begin{proof}
    Es reicht, dies für die Basisvektoren $a = \varphi_{u}(p) \text{ und } b = \varphi_{v}(p)$ zu prüfen!
    Sei $\varphi : U \to \Sigma$ eine $C^{2}$-Parametrisierung mit $p\in\varphi(U)$.
    Betrachte die Komposition $N \circ \varphi : U \to S^{2}$.
    $\forall q = (u,v)\in U$ gilt: $\langle N\circ \varphi(u,v), \varphi_{u}(u,v)\rangle_{\varphi(u,v)} \stackrel{\text{Skalarprodukt in }\mathbb{R}^{3}}{=} 0$ (*)
    \ctikzfig{1 zeichnung mit phiU und phiV usw}
    \begin{notation}
        $N_{u}(u,v) = (DN)_{\varphi(u,v)}(\varphi_{u}(u,v)) = \frac{d}{du}(N\circ\varphi)(u,v)$ und \\
        $N_{v}(u,v) = (DN)_{\varphi(u,v)}(\varphi_{v}(u,v)) = \frac{d}{dv}(N\circ\varphi)(u,v)$ \\
        $\frac{d}{du}(*)_{v} : \langle N_{u}, \varphi_{v} \rangle + \langle N, \underbrace{\varphi_{uv}}_{\frac{d}{du}\varphi_{v}(\varphi \text{ ist } C^{2})}\rangle = 0$\\
        $\frac{d}{dv} (*)_{u} : \langle N_{v}, \varphi_{u} \rangle + \langle N, \varphi_{vu} \rangle = 0$\\
        $\varphi$ ist $C^{2} \implies \varphi_{uv} = \varphi_{vu}\\
        \implies \langle N_{u}, \varphi_{v}\rangle = \langle N_{v}, \varphi_{u}\rangle$ \\ausgeschrieben: $\langle (DN)_{\varphi(u,v)}(\varphi_{u}(u,v)), \varphi_{v}(u,v)\rangle = \langle (DN)_{\varphi(u,v)}(\varphi_{v}(u,v)), \varphi_{u}(u,v)\rangle \\= \langle (DN)_{\varphi(u,v)}(\varphi_{u}(u,v)), \varphi_{v}(u,v)\rangle$
    \end{notation}
\end{proof}
\begin{remark}
    Im Beweis haben wir die Annahme $(\varphi: U \to \Sigma$ ist $C^{2})$ benutzt: $\varphi_{uv}$ ist vorgekommen. 
    Diese Anname ist essenziell, damit $N:\varphi(U) \to S^{2}$ differenzierbar ist. Tatsächlich gilt $N(\varphi(u,v)) = \pm \frac{\varphi_{u}(u,v)\times\varphi_{v}(u,v)}{\|\varphi_u \times \varphi_{v}\|}$ Wir benutzen, dass $\varphi_{u}$ und $\varphi_{v}$ differenzierbar sind, dass heisst $\varphi : U \to \Sigma$ ist zweimal differenzierbar.
\end{remark}
\begin{example}
    Sei $\begin{aligned}
        f:\mathbb{R} &\to \mathbb{R}\\
        x &\mapsto \left\{
            \begin{tabular}{@{}c@{}}
            $0 \text{, falls } x\le 0 $\\
            $x^{2}\text{, falls } x\ge 0$
            \end{tabular}
        \right.
    \end{aligned}$
    \\$f$ ist differenzierbar, aber $f'$ ist bei $x=0$ \emph{nicht} differenzierbar.\\
    Betrachte die Fläche $\Sigma = \{(x,y,z)\in \mathbb{R}_{3} | z = f(x)\} "= \Gamma f\times \mathbb{R}"$, welche die globale $C^{1}$-Parametrisierung $\begin{aligned}
        \varphi : \mathbb{R}^{2} &\to \Sigma\\
        (u,v) &\mapsto (u,v,f(u)) 
    \end{aligned}$ besitzt.\\
    Berechne $\varphi_{u} = (1, 0, f'(u)), \varphi_{v} = (0,1,0),$ und $N(u,v) = \frac{\varphi_{u} \times \varphi_{v}}{\lVert \varphi_{u} \times \varphi_{v} \rVert} = \frac{(-f'(u),0,1)}{\sqrt[]{1 + f'(u)^{2}}} $\\
    Für $u = 0$ gilt $N(u,v) = (0,0,1) = e_{3} $ und für $u \geq 0$ gilt $N(u,v) = \frac{1}{\sqrt[]{1 + 4u^{2}}}(-2,0,1)$
    Versuch, $\frac{d}{du}N(0,0)$ zu berechen:\\
    \begin{enumerate}
        \item $\lim\limits_{\epsilon \rightarrow 0, \epsilon < 0}{\frac{1}{\epsilon}(\underbrace{N(\epsilon)}_{= e_{3}} - \underbrace{N(0,0)}_{= e_{3}}) = 0}$
        \item $\lim\limits_{\epsilon \rightarrow 0, \epsilon > 0}{\frac{1}{\epsilon}(N(\epsilon,0) - N(0,0))} = \lim\limits_{\epsilon \rightarrow 0, \epsilon > 0}{\frac{1}{\epsilon}(-\frac{2\epsilon}{\sqrt[]{1+4\epsilon^{2}}},0,\frac{1}{\sqrt[]{1+4\epsilon^{2}}}-1)} = (2, 0, \dots) \neq e_{3}$
    \end{enumerate}
    Im 2. Punk wird genutzt, dass $\sqrt[]{1+x} \approx \sqrt[]{1+\frac{x}{2}}$, somit $\frac{-2\epsilon}{1+4\epsilon^{2}} \underbrace{\approx}_{\frac{1}{1+x} \approx 1-x} -2\epsilon(1-2\epsilon^{2}) \approx -2$
    Also ist $N(u,v)$ an der Stelle $(0,0)$ nicht differenzierbar!\\
    \emph{Hypothese:} Ab jetzt besitzen alle regulären Flächen mindestens lokale \emph{$C^{1}$}-Parametrisierungen.
\end{example}

\begin{corollary}
    $(DN)_{p}:T_{p}N \rightarrow T_{p}\Sigma$ lässt sich mit einem orthogonalen Koordinatenwelchel diagonalisieren. \\D.h. Die Weingartenabbildung $(DN)_{p}$ hat zwei orthogonale Eigenvektoren $v_{1},v_{2} \in T_{p}\Sigma$ zu reellen Eigenwerten $\lambda_{1}, \lambda_{2} \in \mathbb{R} $.
    Es gilt \begin{align*}
        &K(p) = \det((DN)_{p}) = \lambda_{1} \cdot \lambda_{2}\\
        &H(p) = \frac{1}{2}Spur((DN)_{p}) = \frac{1}{2}(\lambda_{1} \cdot \lambda_{2})
    \end{align*}
\end{corollary}
\begin{definition}
    Die von den Eigenvektoren $v_{1}, v_{2} \in T_{p}\Sigma $ aufgespannten Richtungen heissen \emph{Hauptkrümmungsrichtungen}. 
    Eine $C^{1}$-Kurve $\gamma:(a,b) \rightarrow \Sigma $ heisst \emph{Krümmungslinie}, falls 
    $\forall t \in (a,b)$ gilt $\dot{\gamma}(t)\in T_{\gamma(t)}\Sigma $ ist ein Eigenvektor der Weingartenabbildung $(DN)_{\gamma(t)}:T_{\gamma(t)}\Sigma_{\gamma(t)} $.
\end{definition}
\begin{examples}
    \begin{enumerate} \leavevmode
        \item $E = \mathbb{R}^{2} \times \{0\} \subset \mathbb{R}^{3}$\\
        Hier gilt $\forall p \in E: (DN)_{p} = 0$ also sind die Hauptkrümmungsrichtungen nicht wohldefiniert. Alle Geraden in E sind Krümmungslinien. (Sogar alle $C^{1}$-Kurven $\gamma:\mathbb{R}\rightarrow E$)
        \item $Z = {(x,y,z)\in \mathbb{R}^{3} \ \vline \  x^{2} + y^{2} = 1}$\\
        Hier gilt $\forall p \in Z: (DN)_{p}(e_{3}) = 0 \implies $ alle (vertikalen) Mantellinien in Z sind Krümmungslinien.
        Mehr noch: Bezüglich der Basis ${e_{3}, -y\cdot e_{1} + x\cdot e_{2}}$ von $T_{\underbrace{(x,y,z)}_{=p}}\Sigma$ hat $(DN)_{p}$ die Matrix $\begin{pmatrix} 0 & 0 \\ 0 & 1 \end{pmatrix}$.
        Wir erhalten also eine zweite Schar von Krümmungslinien: horizontale Kreise.
        In Punkten mit $(DN)_{p} \neq \lambda \cdot Id_{T_{p}\Sigma}$ stehen die Krümmungslinien \emph{senkrecht} aufeinander.
    \end{enumerate}
\end{examples}

\section{Die zweite Fundamentalform}
Ziel der nächsten beiden Abschnitte:
\begin{align*}
    K(p) = \frac{\det(II_{p})}{\det(I_{p})}
\end{align*}
wobei $I_{p}, II_{p}$ die erste, bzw. die zweite Fundamentalform ist.
\begin{motivation}
    Sei $\Sigma \subset \mathbb{R}_{3}$ eine $C^{2}$-reguläre Fläche, und $\alpha:(-\epsilon, \epsilon)\rightarrow \Sigma$ eine $C^{2}$-Kurve mit $\alpha(0) = p \in \Sigma$ und $\dot{\alpha}(0) = v \in T_{p}\Sigma$. 
\end{motivation}
\begin{proposition}
    (Satz von Mensier) Sei $N: V \rightarrow S^{2}$ ein lokales Einheitsnormalenfeld $(p\in V\subset\Sigma)$. Dann gilt $\langle \ddot{\alpha}(0), N(p)\rangle = -\langle v, (DN)_{p}(v)\rangle$. Insbesondere hängt die \emph{normale} Beschleunigungskomponente $\langle\ddot{\alpha}(0), N(p)\rangle$ nur von $p = \alpha(0)$ und $v = \dot{\alpha}(0)$ ab.
\end{proposition}
\begin{proof}
    $\forall t \in (-\epsilon,\epsilon)$ gilt:
    \begin{align*}
        &\langle \dot{\alpha}(t), N(\alpha(t))\rangle = 0 \text{ (per Definition von } T_{p}\Sigma \text{ und } N(p))\\
        &\text{Ableiten nach t: } \langle \ddot{\alpha}(t),N(\alpha(t))\rangle + \langle \dot{\alpha}(t), \underbrace{\frac{d}{dt}N(\alpha(t))}_{(DN)_{\alpha(t)}(\dot{\alpha}(t))} \rangle = 0\\
        &\text{Für } t=0 \text{ erhalten wir } \langle\ddot{\alpha}(0), N(p)\rangle = -\langle v, (DN)_{p}(v)\rangle.
    \end{align*}
\end{proof}
\begin{definition}
    Die \emph{zweite Fundamentalform} von $\Sigma$ an der Stelle $p$ ist die Abbildung
    \begin{align*}
        II_{p}:T_{p}\Sigma &\rightarrow \mathbb{R}\\
        v &\mapsto -\langle v, (DN)_{p}(v)\rangle 
    \end{align*}
\end{definition}
\begin{remark}
    Das Vorzeichen von $II_{p}$ hängt von $N$ ab. Falls $\Sigma \subset \mathbb{R}^{3}$ orientierbar ist, 
    dann lässt sich eine globale Wahl von $N$ fixieren (Wahl der Orientierung).
\end{remark}
\begin{recall}
    Die erste Fundamentalform, $I_{p}(v) = \langle v,v\rangle_{p}$ hat bezüglich jeder lokalen Parametrisierung $\varphi:U\rightarrow\Sigma$ eine Matrix
    $\begin{pmatrix} E & F \\ F & G \end{pmatrix}$, bezüglich der Basis ${\varphi_{u}, \varphi_{v}}$ von $T_{p}\Sigma$. 
    $E = \langle \varphi_{u}, \varphi_{u}\rangle, F = \langle \varphi_{u}, \varphi_{v}\rangle, G = \langle \varphi_{v}, \varphi_{v}\rangle$.
\end{recall}
Koeffizienten für $II_{p}$:
\begin{itemize}
    \item $e = -\langle \varphi_{u}, (DN)_{\varphi(u,v)}(\varphi_{u})\rangle$
    \item $f = -\langle \varphi_{u}, (DN)_{\varphi(u,v)}(\varphi_{v})\rangle = -\langle \varphi_{v}, (DN)_{\varphi(u,v)}(\varphi_{u})\rangle$
    $(DN)_{\varphi(u,v)}$ ist symmetrisch.
    \item $e = -\langle \varphi_{v}, (DN)_{\varphi(u,v)}(\varphi_{v})\rangle$
\end{itemize}
\begin{notation}
    $$N_{u} = \frac{d}{du}(N\circ\varphi) = (DN)_{\varphi(u,v)}(\varphi_{u})$$\\
    $$N_{v} = \frac{d}{dv}(N\circ\varphi) = (DN)_{\varphi(u,v)}(\varphi_{v})$$\\
\end{notation}
Wir erhalten also mit $\langle \varphi_{u}, N\rangle \equiv 0$ und ableiten nach w: $\langle \varphi_{uu}, N\rangle + \langle \varphi_{u}, N_{u}\rangle = 0$, analog für $v$ 
\begin{itemize}
    \item $e = -\langle \varphi_{u}, N_{u} \rangle = \langle \varphi_{uu}, N \rangle$
    \item $e = -\langle \varphi_{u}, N_{v} \rangle = \langle \varphi_{uv}, N \rangle$
    \item $e = -\langle \varphi_{v}, N_{v} \rangle = \langle \varphi_{vv}, N \rangle$
\end{itemize}
\begin{example}
    Funktionsgraph von $f: \mathbb{R}^{2} \rightarrow \mathbb{R}$ $(C^{2})$\\
    $\Gamma_{f} = {(x,y,z)\in\mathbb{R}^{3} \ \vline \ z = f(x,y)} \subset \mathbb{R}^{3}$ Globale $C^{2}$-Parametrisierung \begin{align*}
        \varphi:\mathbb{R}^{2} &\rightarrow \Gamma_{f}\\
        (u,v) &\mapsto (u,v,f(u,v))
    \end{align*}
    Berechne 
    \begin{align*}
        &\varphi_{u}(u,v) = (1, 0, f_{u}(u,v))\\
        &\varphi_{v}(u,v) = (0, 1, f_{v}(u,v))\\
        &\varphi_{uu}(u,v) = (0, 0, f_{uu}(u,v))\\
        &\varphi_{uv}(u,v) = \varphi_{vu}(u,v) = (0, 0, \underbrace{f_{uv}}_{=f_{vu}})\\
        &\varphi_{vv}(u,v) = (0, 0, f_{vv}(u,v))\\
        &N(\varphi(u,v)) = \frac{\varphi_{u}\times\varphi_{v}}{\lvert\varphi_{u}\times\varphi_{v}\rvert} = \frac{(-f_{u}, -f_{v}, 1)}{\sqrt[]{1 + f_{u}^{2}+f_{v}^{2}}}\\
        &e = \langle \varphi_{uu}, N \rangle = \frac{f_{uu}}{\sqrt[]{1 + f_{u}^{2}+f_{v}^{2}}}\\
        &f = \langle \varphi_{uv}, N \rangle = \frac{f_{uv}}{\sqrt[]{1 + f_{u}^{2}+f_{v}^{2}}}\\
        &g = \langle \varphi_{vv}, N \rangle = \frac{f_{vv}}{\sqrt[]{1 + f_{u}^{2}+f_{v}^{2}}}\\
    \end{align*}
\end{example}

\section{Gaussabbildung in lokalen Koordinaten}



\subsection*{Rotationsflächen}

\section{Theorema Egregium}



\end{document}