\documentclass[../main.tex]{subfiles}

\begin{document}

\chapter{Ebene hyperbolische Geometrie}
Knörrer: Geometrie Kapitel 3

\begin{goal}
    Konstruktion einer \emph{vollständigen} Fläche $H$ mit konstanter Krümmung $-1$, analog zur Ebene
    ($K \equiv 0$) und Sphäre ($K \equiv 1$).\\
    \emph{Vollständig}: Jede geodätische Kurve $\gamma : (a,b) \to H$ lässt sich geodätisch auf $\mathbb{R}$ erweitern.
\end{goal}

\begin{motivation}
    \emph{Gauss-Bonnet}
    $$\int _ \Sigma K \ dA = 2 \pi \chi (\Sigma)$$
    wobei $\Sigma$ eine kompakte, vollständige Fläche.
    Falls $\chi (\Sigma) < 0$ und die Krümmung $K$ konstant ist, dann muss $K$ negativ sein!

    \begin{theorem}[Klassifikation der Flächen]
        Sei $\Sigma$ eine topologische (glatte), kompakte, vollständige, orientierbare, zusammenhängende Fläche.
        Dann ist $\Sigma$ zu einer der Flächen $\Sigma _g$ homöomorph (diffeomorph): \\
        \begin{figure}[htb]
            \centering
            \def\svgwidth{20em}
            \input{figures/henkel.pdf_tex}
            \caption*{$\Sigma _g$ mit $g$ Henkel}        
        \end{figure}

        Es gilt: $\chi (\Sigma _g)= 2-2g <0$ falls $g \ge 2$.

    \end{theorem}
\end{motivation}

\newpage
\section{Eine Riemannsche Metrik mit K=-1}
Naiver Ansatz zur Konstruktion einer Riemannschen Metrik auf $\mathbb{R}^2$ mit $K=-1$.
$$\langle \ , \ \rangle _p = h(p) \langle \ , \ \rangle_{\mathbb{R}^2}$$
wobei $h:\mathbb{R}^2 \to \mathbb{R}$ positiv und glatt ist. Für die Koeffizientenfunktionen $E, F, G$ gilt
also: 
\begin{itemize}
    \item $E(x,y) = \langle e_1, e_1 \rangle _{(x,y)} = h(x,y)$
    \item $F(x,y) = \langle e_1, e_2 \rangle _{(x,y)} = 0$
    \item $G(x,y) = \langle e_2, e_2 \rangle _{(x,y)} = h(x,y)$
\end{itemize}
\begin{terminology}
    Falls $E=G$ und $F=0$ gilt, dann heissen die Koordinaten \emph{konform} oder \emph{isotherm}.
\end{terminology}
Eine kleine Rechnung zeigt
$$K = - \frac{1}{2 h(x,y)} \Delta (\log (h(x,y)))$$ wobei $\Delta f = f_{xx} + f_{yy}$ der Laplaceoperator (siehe Serie 10).
\\
Nun führt $K=-1$ zu einer Differentialgleichung für $h$:
$$2 h(x,y)=\Delta (\log (h(x,y)))$$
Dies ist eine \emph{partielle Differentialgleichung}, welche schwierig zu lösen ist. Mit dem Lösungsansatz $h(x,y)=y^n$ finden wir eine Lösung
$h(x,y)=\frac{1}{y^2}$, welche allerdings nur auf der obenen Halbebene
$H= \{ z = x+ iy \in \mathbb{C} \ \vert \ y > 0 \}$ definiert ist.

\begin{definition}
    Die \emph{hyperbolische Ebene} ist die Menge $H= \{ z \in \mathbb{C} \ \vert \ \im (z) > 0 \}$ mit
    der Riemannschen Metrik
    $$\langle \ , \ \rangle _{x+iy} = \frac{1}{y^2} \langle \ , \ \rangle _{\mathbb{R}^2}$$
\end{definition}

\begin{remarks}
    \leavevmode
    \begin{enumerate}
        \item Die Translation $z \mapsto z + a$ mit $a \in \mathbb{R}$ ist eine Isometrie von $H$.
        Tatsächlich, schreibe
        \begin{align*}
            T : H & \to H \\ (x,y) & \mapsto (x+a,y)
        \end{align*}
        Für alle $p \in H$ gilt $(DT)_p = Id_{\mathbb{R}^2}$. Zu prüfen für alle $v,w \in \mathbb{R}^2$:
        $$\langle v,w \rangle _p \overset{?}{=} \langle (DT)_p(v),(DT)_p(w) \rangle _{T(p)} = \langle v,w \rangle _{T(p)}$$
        Stimmt, da $y(p)=y(T(p))$, und somit $\langle \ , \ \rangle _p = \langle \ , \ \rangle _{T(p)}$

        \item Die Streckung $z \mapsto \lambda z$ mit $\lambda >0$ ist eine Isometrie von $H$.
        Schreibe
        \begin{align*}
            S : H & \to H \\ (x,y) & \mapsto (\lambda x, \lambda y)
        \end{align*}
        Für alle $p \in H$ gilt $(DS)_p = \lambda Id_{\mathbb{R}^2}$. Zu prüfen für alle $v,w \in \mathbb{R}^2$:
        $$\langle v,w \rangle _p \overset{?}{=} \langle (DS)_p(v),(DS)_p(w) \rangle _{S(p)} = \lambda ^2 \langle v,w \rangle _{S(p)}$$
        Stimmt, da $y(S(p)) = \lambda y(p)$, also $\langle \ , \ \rangle _{S(p)} = \frac{1}{\lambda ^2} \langle \ ,\ \rangle _p$

        \item Die Inversion $z \mapsto -\frac{1}{z}$ ist eine Isometrie von $H$.
        Schreibe $$\varphi (z) = -\frac{1}{z} = - \frac{\bar{z}}{|z|^2}=\frac{-x+iy}{x^2+y^2} \in H$$ falls $z\in H$ d.h. $y>0$.
        Also $\varphi : H \to H$. Es gilt für alle $z \in H$ und $v \in \mathbb{C} = \mathbb{R}^2$:
        $$(D\varphi )_z (v)=\varphi ' (z) v = -\frac{1}{z^2}v$$
        Zu prüfen: $$\langle v, w \rangle _z \overset{?}{=} \langle -\frac{1}{z^2}v,-\frac{1}{z^2}w \rangle _{-\frac{1}{z}} = \frac{1}{|z|^4}\langle v,w\rangle _{-\frac{1}{z}}$$.
        Stimmt, da $y(-\frac{1}{z}) = \frac{1}{|z|^2} y(z) \implies \langle \ , \ \rangle _{-\frac{1}{z}} = |z|^4 \langle \ , \ \rangle _{z}$ 

    \end{enumerate}
\end{remarks}

\section{Möbiustransformationen}
\begin{recall}[aus der komplexen Analysis]
    Für $\begin{pmatrix}
        a & b \\ c & d
    \end{pmatrix} \in GL(\mathbb{C}^2)$, definieren wir die zugehörige \emph{Möbiustransformation} (nicht auf ganz $\mathbb{C}$ definiert).
    \begin{align*}
        (MT) \Phi : \mathbb{C} & \dashrightarrow \mathbb{C} \\
            z & \mapsto \frac{az+b}{cz+d}
    \end{align*}

\end{recall}

\begin{examples}
    \leavevmode
    \begin{enumerate}
        \item $A = \begin{pmatrix}
            1 & b \\ 0 & 1
        \end{pmatrix} b \in \mathbb{C} \implies \Phi _A (z)=z+b$

        \item $A = \begin{pmatrix}
            a & 0 \\ 0 & a^{-1}
        \end{pmatrix} a \in \mathbb{C}\setminus \{0\} \implies  \Phi _A (z)=a^2z$.
        Für $a = \sqrt{\lambda}: \lambda z \ (\lambda > 0)$

        \item $A = \begin{pmatrix}
            0 & 1 \\ -1 & 0
        \end{pmatrix} \implies \Phi _A (z)= -\frac{1}{z}$
        Insbesondere, für $a = \sqrt{\lambda} (\lambda > 0): \lambda z$
    \end{enumerate}
\end{examples}

\begin{remark}
    Die obigen Isometrien 1-3 sind vom Typ $\Phi _A$ mit $A \in SL(\mathbb{R}^3)$.
    (Determinante $1$)
\end{remark}

\subsection*{Projektive Interpretation von Möbiustransformation}
Sei $A \in GL(\mathbb{C}^2)$. Dann erhalten wir eine lineare Abbildung
$A : \mathbb{C}^2 \to \mathbb{C}^2$. Insbesondere bildet $A$ Geraden durch 0 auf
Geraden durch 0 ab (1).
\begin{definition}
    Die \emph{projektive Gerade} $\mathbb{P}(\mathbb{C}^2) = \mathbb{P}^1\mathbb{C}$ ist die Menge
    aller komplexen Geraden durch 0 in $\mathbb{C}^2$. Konkret:
    Die Menge der Äquivalenzklassen bezüglich folgender Äquivalenzrelation auf $\mathbb{C}^2 \setminus \{0\}$:
    $$v \sim w \iff \exists \lambda \in \mathbb{C}, \lambda \not = 0 \text{ mit }w = \lambda v$$
\end{definition}

Dann ist 
$\mathbb{P}(\mathbb{C}^2) = \big (\mathbb{C}^2 \setminus \{0\} \big ) / \sim$.
Sei nun $\begin{pmatrix}
    a \\ b
\end{pmatrix} \in \mathbb{C}^2 \setminus \{0\}$
\begin{itemize}
    \item Falls $b \not = 0$, dann gilt 
    $v=\begin{pmatrix}
        a \\b
    \end{pmatrix} \sim \begin{pmatrix}
    \frac{a}{b} \\1
    \end{pmatrix} = \begin{pmatrix}
        z \\1
        \end{pmatrix}
    z \in \mathbb{C}$.

    \item Falls $b=0$, dann gilt
    $v=\begin{pmatrix}
        a \\b
    \end{pmatrix} \underset{a \not = 0}{\sim} \begin{pmatrix}
    1 \\ 0
    \end{pmatrix} = \infty$.

\end{itemize}
Daraus folgern wir, dass $\mathbb{P}(\mathbb{C}^2) = \mathbb{C} \cup \{ \infty \}$.
Aus (1) folgt: Die Abbildung $A : \mathbb{C}^2 \to \mathbb{C}^2$ induziert eine Abbildung
\begin{align*}
    \Phi _A : \mathbb{P}(\mathbb{C}^2) & \to \mathbb{P}(\mathbb{C}^2) \\
    [v] & \mapsto [Av]
\end{align*}
Interpretation via $\mathbb{P}(\mathbb{C}^2) = \mathbb{C} \cup  \{ \infty \}$.

\begin{itemize}
    \item $v = \begin{pmatrix}
        z \\ 1
    \end{pmatrix} \implies Av = \begin{pmatrix}a&b\\c&d\end{pmatrix}\begin{pmatrix}z \\ 1\end{pmatrix}=\begin{pmatrix}az+b\\cz+d\end{pmatrix}
    \sim \begin{pmatrix}\frac{az+b}{cz+d} \\ 1\end{pmatrix}\text{ bzw. } cz+d=0 \sim \begin{pmatrix}1\\0\end{pmatrix}$\nolinebreak

    \item $v = \begin{pmatrix}1 \\ 0 \end{pmatrix} \implies Av = \begin{pmatrix}a&b\\c&d\end{pmatrix}\begin{pmatrix}1 \\ 0\end{pmatrix}=\begin{pmatrix}a \\ c \end{pmatrix}
    \sim \begin{cases}
        \begin{pmatrix}\frac{a}{c}\\1 \end{pmatrix}\text{ falls } c \not = 0 \\
        \begin{pmatrix}1\\0 \end{pmatrix} \text{ falls } c = 0
            
    \end{cases}$
\end{itemize}

Notation für $\Phi _A : \Phi _A(z)= \dfrac{az+b}{cz+d}$
``geeignet interpretiert''.
Aus dieser Definition folgt auch dass $\Phi _{AB} = \Phi _A \circ \Phi _B$. Diese Tatsache ist mit der anderen Definition
mühsam zu beweisen.

\begin{lemma}
    Sei $A =\begin{pmatrix}
        a&b\\c&d
    \end{pmatrix}\in SL(\mathbb{R}^2)$
    Dann erhält die Möbiustransformation $\varphi _A$ die obere Halbebene
    $H=\{ z\in \mathbb{C} \ | \ \im (z)>0\}$.
\end{lemma}

\begin{proof}
    Sei $z \in H$, d.h. $\im (z)= \frac{1}{2i} (z- \bar{z}) >0$.
    Berechne
    \begin{align*}
        \im(\varphi _A (z)) &= \frac{1}{2i}\left( \frac{az+b}{cz+b}-\frac{a\bar{z}+b}{c\bar{z}+d} \right) \\
        &= \frac{1}{2i} \frac{(az+b)(c\bar{z}+d)+(a\bar{z}+b)(cz+d)}{|cz+d|^2} \\
        &= \frac{1}{2i} \frac{(ad-bc)(z-\bar{z})}{|cz+d|^2} \\
        & \overset{det A=1}{=} \frac{\im (z)}{(cz+d)^2} >0
    \end{align*}
\end{proof}
\begin{remark}
    Es gilt sogar $\varphi _A(H) = H$
    Tatsächlich gilt 
    \begin{align*}
        H=\varphi _{\begin{pmatrix}
            1 & 0 \\ 0  & 1
        \end{pmatrix}}(H) &= \varphi _{A \circ A^{-1}}(H) \\
        &\overset{\varphi_{AB}=\varphi_A \circ \varphi_B}{=} \varphi _A \circ \varphi _{A^{-1}}(H)=\varphi_A (\varphi _{A^{-1}}(H))\subset \varphi _A (H)
    \end{align*} $\implies \varphi _A (H)=H$. Daraus folgt, dass Möbiustransformationen eine Gruppe bilden.
    
\end{remark}

\begin{lemma}
    Jede Möbiustransformation $\varphi _A : H \to H$ mit $A \in SL(\mathbb{R}^2)$ ist eine endliche
    Komposition von Möbiustransformationen der Form
    \begin{enumerate}
        \item $z \mapsto z + b \qquad (b\in \mathbb{R})$ horizontale Translation
        \item $z \mapsto \lambda z \qquad (\lambda >0)$ Streckung
        \item $z \mapsto -\frac{1}{z} $ Inversion
    \end{enumerate}
\end{lemma}

\begin{proof}
    \begin{align*}
        \frac{az +b}{cz+d} =\frac{a}{c} \frac{cz+\frac{c}{a}b}{cz +d}=\frac{a}{c}\frac{cz+d+(\frac{c}{a}b-d)}{cz+d}
        = \alpha + \frac{\beta}{cz+d}
    \end{align*} für geeignete $\alpha$ und $\beta$.
    Details siehe Serie 11.
\end{proof}
\begin{corollary}
    Alle Möbiustransformationen der Form $\varphi _A : H \to H$ mit $A \in SL(\mathbb{R}^2)$
    sind Isometrien bezüglich der Riemannschen Metrik $\frac{1}{y^2}\langle \ , \ \rangle _{\mathbb{R}^2}$.
\end{corollary}
\begin{proof}
    Möbiustransformationen des Typs 1-3 sind Isometrien, siehe oben
\end{proof}
\begin{lemma}
    Die Kurve $\begin{aligned}[t]
        \gamma : \mathbb{R} & \to H \\
        t & \mapsto i e^t
    \end{aligned}$
\end{lemma}

\begin{proof}
    Wir bemerken zuerst, dass
    \begin{align*}
        || \dot{\gamma}(t)||_H &= \sqrt{\langle \dot{\gamma}(t), \dot{\gamma}(t)\rangle _H} \\
        &\overset{y(\gamma(t))=e^t}{=} \sqrt{\frac{1}{(e^t)^2} \underbrace{\langle i e^t, i e ^t \rangle _{\mathbb{R}^2}}_{\langle e^t, e^t \rangle = (e^t)^2}}=1
    \end{align*}
    $\implies \gamma :\mathbb{R}\to H $ ist nach Bogenlänge parametrisiert (bzgl. hyperbolischer Metrik).
    Sei nun $\delta : \mathbb{R} \to H$ die eindeutige geodätische Kurve mit
    $\delta (0) =i$ und $\dot{\delta}(0)=i$.
    Betrache die folgende Isometrie von $H$ (Spiegelung an $i \mathbb{R})$
    \begin{align*}
        \sigma : H & \to H \\
        x +iy & \mapsto -x + iy
    \end{align*}
    Nun ist $\sigma \circ \delta :\mathbb{R}\to H$ auch geodätisch mit $\sigma \circ \delta(0)=i$
    und auch $\frac{d}{dt}(\sigma \circ \delta)(0)=i$.
    Aus der Eindeutigkeit der Geodäten zu Anfangsbedingungen folgt also $\delta = \sigma \circ \delta$,
    also $\delta (\mathbb{R}) \subset i \mathbb{R}$.
    Da $\gamma$ und $\delta $ nach Bogenlänge parametrisiert ($\delta$ ist geodätisch mit $||\dot{\delta}(0)||_H =1$!) sind, folgt $\gamma = \delta$. 
\end{proof}

\begin{proposition}
    Die Geodäten in $H$ sind genau die Halbgeraden und Halbkreise, welche senkrecht auf ``$\mathbb{R} \cup \{\infty \} = \partial H$''.
    stehen. 
\end{proposition}

\begin{figure}[htb]
    \centering
    \def\svgwidth{20em}
    \input{figures/proposition_circles.pdf_tex}
    \caption*{Geodäten in Halbebene H}        
\end{figure}

\begin{proof}
    Wir haben schon eine Geodäte gefunden:
    $i \mathbb{R}_{>0} $, das Bild der Kurve $\gamma (t) = ie^t$.
    Schreibe $h = \bild (\gamma) \subset H$. Nun ist für jede Isometrie
    $\varphi : H \to H, \varphi (H) \subset H$ auch eine Geodäte. Insbesondere
    können wir auf $h$ iteriert Abbildung der Form
    \begin{enumerate}
        \item $z \mapsto z + b \qquad b \in \mathbb{R}$
        \item $z \mapsto \lambda z \qquad \lambda > 0$
        \item $z \mapsto -\frac{1}{z}$
    \end{enumerate}
    Daraus folgt, dass alle Halbgeraden auf $\mathbb{R}$ Geodäten sind.
    Betrachte die spezielle Isometrie $\varphi (z)=-\frac{2}{z+1}$
    \begin{claim}
        $\varphi (h)$ ist ein Halbkreis in $H$ mit Zentrum $-1$ und Radius $1$
    \end{claim}
    \begin{proof}
        Sei $i y \in h$. Berechne
        \begin{align*}
            |\varphi(iy)+1| &= \left| -\frac{2}{iy+1} + \frac{iy+1}{iy+1} \right| = \left| \frac{iy -1}{iy+1} \right| = 1
        \end{align*}\end{proof}
    Unter Anwendung von horizontalen Transformationen und Streckungen erhalten wir aus $\varphi(h)$
    alle Halbkreise Senkrecht auf $\mathbb{R}$.
    \begin{figure}[htb]
        \centering
        \def\svgwidth{20em}
        \input{figures/geodesic_halfcircle.pdf_tex}
        \caption*{Halbkreis als Geodäte}        
    \end{figure}

    \begin{question}
        Wieso existieren keine weiteren Geodäten?
    \end{question}
    Zu jeden $z \in H$ und jedem Einheitsvektor $v$ existiert genau eine geodätische Kurve $\gamma : \mathbb{R} \to H$
    mit $\gamma (0) = z$ und $\dot{\gamma} (0) = v$. Das Bild von $\gamma$ muss also der Halbkreis
    oder die Halbgerade durch $z$ mit Tangente $v$ sein!
    \begin{figure}[htb]
        \centering
        \def\svgwidth{10em}
        \input{figures/geodesic_uniqueness.pdf_tex}
        \caption*{Eindeutigkeit der Geodäte}        
    \end{figure}
\end{proof}

\begin{remark}
    Für alle $z, w \not = z \in H$ existiert eine Geodäte $g < H$ mit $z,w \in g$
    Hingegen existiert zu $g \subset H$ und $z \not \in g$ unendlich viele Geodäten $h \in H$ mit $z \in h$ und $h \cap g = \emptyset$.
    Wir bemerken, dass das \emph{Parallelaxiom} in der hyperbolischen Ebene \emph{nicht erfüllt} ist.
    \begin{figure}[htb]
        \centering
        \def\svgwidth{30em}
        \input{figures/geodesic_remark.pdf_tex}        
    \end{figure}
\end{remark}
Dies wurde etwa 1840 von Bolyai und Lobachevski bemerkt.

\section{Die Isometriegruppe von H}
\begin{lemma}
    Sei $\varphi : H \to H$ eine orientierungserhaltende Isometrie,
    d.h. für alle $z \in H$ gilt $\det \left ((D\varphi)_z > 0 \right )$. Dann ist $\varphi$ durch
    $\varphi (i) \in H$ und $(D\varphi)_i(i) \in \mathbb{C}$ eindeutig bestimmt.
\end{lemma}
\begin{proof}
    Geometrisch, unter Benutzung der Tatsache, dass Isometrien winkelerhaltend sind. Wir bemerken zuerst,
    dass $\delta (t) = \varphi (ie^t)$ eine geodätische Kurve mit $\delta (0) = \varphi (i)$ und
    $\dot {\delta}(0) = (D\varphi)_{ie^0}(i)= (D\varphi)_i(i)$ ist, also durch $\varphi (i) \in H$ und $(D\varphi)_i(i) \in \mathbb{C}$ bestimmt.
    Insbesondere kennen wir auch $\varphi (2i)\in H$.
    Sei $z \in H \setminus i\mathbb{R}_{>0}, g_i, g_z \subset H$ Geodäten mit $i,z \in g_i$
    bzw. $2 i, z \in g_z$, wegen $\measuredangle  (g_i,h) = \measuredangle (\varphi (g_i), \varphi (h)) $ und $\varphi$ winkelerhaltend.
    Daraus folgt $\varphi (g_i)$ und $\varphi (g_z) \subset H$ festgelegt. Daraus erhalten wir $\varphi (z) = \varphi (g_i) \cap \varphi (g_z)$.
    Dies ist ein eindeutiger Schnittpunkt, da es Halbkreise senkrecht auf $H$ sind.
    \begin{figure}[htb]
        \centering
        \def\svgwidth{40em}
        \input{figures/lemma.pdf_tex}        
    \end{figure}
\end{proof}

\begin{definition}
    $\Iso ^+ (H) = \{ \varphi : H \to H \ | \ \varphi \text{ ist eine orientierungserhaltende Isometrie}\}$ 
    Dies ist eine Gruppe unter der üblichen Komposition.
\end{definition}
Für alle $A \in SL(\mathbb{R}^2)$ gilt $\varphi _A \in \Iso^+ (H)$.
Wir erhalten also eine Abbildung
\begin{align*}
    \Psi : SL(\mathbb{R}^2) &\to \Iso ^+(H) \\
    A &\mapsto \varphi _A 
\end{align*} welche ein Gruppenhomomorphismus ist: $\varphi _{AB} = \varphi _A \circ \varphi_B$

\begin{theorem}
    $\Psi$ ist surjektiv, es gilt $\ker (\Psi) = \left \{ \pm  E = \pm  \begin{pmatrix}
        1 & 0 \\ 0 & 1
    \end{pmatrix} \right \}$
    Insbesondere gilt $\Iso ^+ (H) \simeq SL(\mathbb{R}^2) / \pm E = PSL(\mathbb{R}^2)$
\end{theorem}

\begin{proof}
    \leavevmode
    \begin{enumerate}
        \item $\ker (\Psi) = \{ \pm E\}$: Sei $A = \begin{pmatrix}
            a & b \\ c & d
        \end{pmatrix} \in SL(\mathbb{R}^2)$ mit $\varphi _A = Id_H$,
        d.h. für alle $z \in H$
        $\frac{az+b}{cz+d}=z$ bzw. $cz^2 + (d-a)z -b=0$.
        Wir folgern $c=0, d=a, b = 0$. Also $A = \begin{pmatrix}
            a & 0 \\ 0 & a
        \end{pmatrix}$, mit $\det (A) = a^2 = 1 \implies a = \pm 1 (A = \pm E)$ 

        \item $\Psi $ ist surjektiv: Sei $\varphi \in \Iso ^+ (H)$. Betrachte
        $\varphi (h) = \varphi (i \mathbb{R}_{>0}) \subset H$. Aus obigen Ausführungen wissen wir,
        dass eine Möbiustransformation $\varphi _A$ existiert mit $\varphi _A (h)=\varphi(h)$.
        Daraus folgt $\underbrace{\varphi _{A^{-1}} \circ \varphi (h)}_{\in \Iso ^+(H)} = h$.
        Nach einer Streckung $\varphi _B(z)=\lambda z$ gilt sogar $\varphi _B^{-1} \circ \varphi _{A^{-1}} \circ \varphi (i) =i$.
        Es kann sein, dass $\varphi _B^{-1} \circ \varphi _{A^{-1}} \circ \varphi$ die Geodäte $h$ um
        $180$° dreht, um den Punkt $i$.
        Entweder ist
        $$\varphi _B^{-1} \circ \varphi _{A^{-1}} \circ \varphi = \begin{cases}
            Id _H \implies \varphi = \varphi _A \circ \varphi _B = \varphi _{AB} = \Psi (AB) \\
            \varphi _{C} \implies \varphi = \varphi _A \circ \varphi _B \circ \varphi _C = \varphi _{ABC}=\Psi (ABC)
        \end{cases}$$
    \end{enumerate}
\end{proof}

\begin{consequence}
    Isometrien von $H$, welche die Orientierung erhalten, sind Möbiustransformationen $\varphi _A$ mit $A \in SL(\mathbb{R}^2)$
\end{consequence}

\section{Distanz und Flächeninhalt}
Seien $p, q \in H$.
\begin{definition}
    $d_H (p,q) = \inf \{L(\gamma) \ | \ \gamma : [a,b] \to H \ C^1 \text{ mit } \gamma (a)=p, \gamma(b)=q \}$
    wobei
    $$L(\gamma) = \int_a^b || \dot{\gamma}(t)||_H \ dt = \int _a^b \sqrt{\langle \dot{\gamma}(t), \dot{\gamma}(t) \rangle _H} \ dt$$
\end{definition}

\begin{lemma}
    Für alle $T \ge 1$ gilt : $d _H (i, Ti) = \log (T)$.
\end{lemma}
\begin{proof}
    Betrachte zuerst die Kurve
    \begin{align*}
        \gamma : [0, \log(T)] & \to H \\
        t & \mapsto ie^t
    \end{align*}
    Es gilt $\gamma(0) = i, \gamma(\log(T))= Ti$. Berechne
    \begin{align*}
        L (\gamma) &= \int _0 ^{\log (T)} \sqrt{\frac{1}{(e^t)^2}\langle ie^t, ie^t \rangle} \ dt \\
        &= \int _0 ^{\log (T)} 1 \ dt = \log (T)
    \end{align*}
    Hier wird benutzt, dass $\langle \ , \ \rangle _H = \frac{1}{y^2} \langle \ , \ \rangle _ {\mathbb{R}^2}$ und $y(\gamma (t))= e^t$.
    \\ \\
    Sei nun $\delta :[a,b] \to H \ C^1$ mit $\delta (a) =i, \delta (b)=T$ ein beliebiger $C^1$-Weg.
    Schreibe $\delta (t) = x(t)+ iy(t)$ mit $\dot{\delta}(t)=\dot{x}(t)+i\dot{y}(t)$.
    Schätze ab:
    \begin{align*}
        L (\delta) &= \int _a ^b \sqrt{\langle \dot{\delta}(t), \dot{\delta}(t) \rangle _H} \ dt \\
        &= \int _a ^b \frac{1}{y(t)} \sqrt{(\dot{x}(t)^2+\dot{y}(t)^2)} \ dt \\
        & \ge \int _a ^b \sqrt{\frac{\dot{y}(t)^2}{y(t)^2}} \ dt \\
        &= \int _a ^b \left | \frac{\dot{y}(t)}{y(t)} \right | \ dt \\
        &\ge \int _a ^b \frac{\dot{y}(t)}{y(t)} \ dt = \log (y(b)) - \log(\underbrace{y(a)}_{=1})= \log(T) - 0
    \end{align*}    
\end{proof}

\begin{proposition}
    Für alle $z,w \in H$ gilt 
    $$\cosh (d_H(z,w))=1 \frac{|z-w|^2}{2 * \im (z) \im(w)}$$
    Zur Erinnerung: $\cosh (x)=\frac{1}{2}(e^x + e^{-x})$
\end{proposition}
\begin{remarks}
    \leavevmode
    \begin{enumerate}
        \item Für $z=w$ gilt $d_H (z,w)=0$, also $\cosh(d_H(z,w))=1$. Deshalb ``$+1$''
        \item Sei $x\in \partial H = \mathbb{R} \cup \{\infty\}$. Dann gilt für festes $z\in H$:
            $$\lim_{w \to x} d_H(z,w) = + \infty$$ da $\im(w)\to 0$.
            ``Punkte im Rand $\partial H$ sind unendlich weit weg''
    \end{enumerate}
\end{remarks}

\begin{proof}
    Seien zunächst $z,w \in i\mathbb{R}_{>0}$: schreibe $z=ia$ und $w=ib$ mit $a<b$, (sonst benutze $d_H(z,w)=d_H(w,z)$).
    Für den Weg $\gamma : [\log (a), \log(b)]\to H$ und $ t \mapsto i e^t$, gilt $\gamma(\log(a))=ia=z, \gamma(\log(b)=ib=w)$,
    und $L(\gamma) = \log (b)-\log(a)$. Für alle anderen Wege $\delta : [c,d]\to H$ mit
    $\delta(c)=z$ und $\delta(d)=w$ gilt:
    \begin{align*}
        L(\delta)= \int _c ^d || \dot{\delta}(t)||_H \ dt & = \int_c ^d \frac{1}{y(t)}\sqrt{\dot{x}(t)^2+\dot{y}(t)^2}\ dt \\
        &\ge \int_c ^d \frac{1}{y(t)}\sqrt(\dot{y}(t)^2)\ dt
    \end{align*}$\implies$
    $$L(\delta) \ge \int_c^d \frac{\dot y (t)}{y(t)} \ dt = \left[ \log (y(t)) \right]_c^d = \log(y(d))-\log(y(c))=\log(b)-\log(a)=\log(\frac{b}{a})$$
    Wir folgern $d_H (ia,ib)= \log(\frac{b}{a})$. Aus allem folgt dann $\cosh(d_H(ia,ib))=\frac{1}{2}(\frac{a}{b}+\frac{a}{b})=\frac{1}{2}\frac{b^2+a^2}{ab}=1+ \frac{(a-b)^2}{2ab}$
    Also folgt die Proposition für $z,w \in i\mathbb{R}_{>0}$. Für den allgemeinen Fall:
    Seien $z \not = w\in H$ beliebig. Dann existiert eine Isometrie $\varphi : H \to H$ (eine Möbiustransformation $\varphi _A : H \to H$ mit $A \in SL(\mathbb{R}^2)$),
    welche die Geodäte durch $z,w$ auf die Geodäte $i \mathbb{R}_{>0}$ abbildet (siehe oben).

    Insbesondere gilt $\varphi (z)=ia$ und $\varphi(w)=ib$. Wir bemerken, dass $$d_H(z,w)=d_H(\varphi(z),\varphi(w))=d_H(ia,ib)$$ gilt (da $\varphi$ eine Isometrie).
    Falls wir zeigen können, dass $\varphi$ auch den Ausdruck
    $$1+\frac{|z-w|^2}{2\im(z)\im(w)}$$ erhält, dann sind wir fertig! Es reicht, dies für Möbiustransformationen des Typs 1 bis 3 zu zeigen.
    \begin{enumerate}
        \item $z \mapsto z+ c \qquad(c\in \mathbb{R})$ invariant, da Differenz
        \item $z \mapsto \lambda z \qquad (\lambda > 0)$ ok, da $|\lambda z - \lambda w |^2 = \lambda^2 |z-w|^2, \im(\lambda a)=\lambda \im(a)$ 
        \item $z \mapsto -\frac{1}{z}$
        Die letzte Transformation ist ok, da
        $$1 + \frac{|-\frac{1}{z}+\frac{1}{w}|^2}{2 \im(-\frac{1}{z})\im(-\frac{1}{w})}=1 + \frac{\frac{|w-z|^2}{|z|^2|w|^2}}{2\frac{\im(z)}{|z|^2}\frac{\im(w)}{|w|^2}}=1+\frac{|w-z|^2}{2\im(z)\im(w)}$$
    \end{enumerate}
\end{proof}
\newpage
\subsection*{Flächeninhalt}
Sei $\Delta \in H$ ein geodätisches Dreieck. Nach Gauss-Bonnet (lokal) gilt:
$$\int_{\Delta}K \ dA = \underbrace{\int _{\Delta} (-1)\ dA}_{area(\Delta)} = \alpha + \beta + \gamma - \pi$$
Also gilt $area(\Delta)=\pi - (\alpha + \beta + \gamma)$

Wir überprüfen dies durch Integration.
\begin{specialcase}
    $\alpha = \beta = \gamma = 0$. Das heisst $\Delta$ ist ein \emph{ideales Dreieck}
    mit Eckpunkten in $\partial H$
\end{specialcase}
\begin{figure}[htb]
    \centering
    \def\svgwidth{10em}
    \input{figures/ideal_triangle.pdf_tex}
    \caption*{Ideales Dreieck mit Winkeln 0}        
\end{figure}

\begin{claim}
    Es existiert eine Isometrie $\varphi : H \to H$, welche die Eckpunkte von $\Delta$ auf $-1,+1,\infty$ schickt!
\end{claim}
\begin{figure}[htb]
    \centering
    \def\svgwidth{35em}
    \input{figures/proof_claim.pdf_tex}
    \caption{Beweis der Behauptung}        
\end{figure}

Berechne nun den Flächeninhalt vom letzten Dreieck $\Delta _0$
\begin{align*}
    area(\Delta)&= \int_{\Delta} \ dA \\
    &= \int _{\Delta}\sqrt{EG-F^2}\ dxdy = \int_{\Delta_0}\left(\frac{1}{y^2}dy \right)dx\\
    &= \int _{-1}^{1}\left( \int \limits_{\sqrt{(1-x^2)}}^{+\infty} \frac{1}{y^2}\ dy\right) dx \\
    &= \int _{-1}^{1} \frac{1}{\sqrt{1-x^2}}\ dx \\
    &= \arcsin(1)-\arcsin(-1) = \frac{\pi}{2}-(-\frac{\pi}{2})=\pi
\end{align*}
Hier wird benutzt $\int \frac{1}{y^2} \ dy = -\frac{1}{y}$.
Ähnlich funktionert dies für ein Dreieck $\Delta _{\alpha}$ mit $\alpha >0$.
$\beta = \gamma = 0$.
$\implies$ $$area(\Delta_\alpha)= \int _{-1}^{\cos(\alpha)}(\int_{\sqrt{-x^2}}^{+\infty}\frac{1}{y^2} \ dy) \ dx = \dots = \arcsin(\cos(\alpha))-\arcsin(-1)$$.
Wir nutzen $cos \alpha = sin(\frac{\pi}{2}-\alpha) \implies area(\Delta _{\alpha})= \pi - \alpha$

Im allgemeinen Fall $alpha, \beta, \gamma >0$ berechnen wir $area(\Delta)$ mit folgendem Ergänzungsbild.

\begin{figure}[htb]
    \centering
    \def\svgwidth{35em}
    \input{figures/general_case.pdf_tex}
    \caption*{Allgemeiner Fall}        
\end{figure}


\section{Ausblick Teichmüllertheorie}
(Nicht mehr Prüfungsrelevant)
\begin{recall}
    Sei $\Sigma _g$ die Standardfläche vom Geschlecht $g \ge 2$. Dann gilt $\chi (\Sigma _g)=2 -2g <0$.F
    Falls auf $\Sigma _g$ eine Metrik mit konstanter Krümmung $K$ existiert, dann muss $K$ \emph{negativ} sein.
    $$\int _{\Sigma _g}K \ dA = 2 \pi \chi(\Sigma _g) < 0$$
\end{recall}
Konstruktion einer Riemannschen Metrik auf $Sigma _g$ mit $K= -1$.
\begin{lemma}
    In $H$ existieren rechtwinklige Sechsecke.
\end{lemma}
\begin{proof}
    Starte mit idealem Seckseck: Ziehe Eckpunkte nach oben bis die Eckpunkte
    rechtwinklig aufeinander sind.
    Alternativer Beweis via Cayleytransformation.
    Verklebe zwei solche Secksecke $S_1$ und $S_2$ entlang dreier Seiten; erhalte eine
    Hose. (dies ist ein abstrakter Prozess, nicht in $\mathbb{R}^3$!).
    Dann verklebe Hosen zu geschlossenen Flächen.
\end{proof}

\end{document}
