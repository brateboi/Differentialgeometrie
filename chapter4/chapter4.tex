\documentclass[../main.tex]{subfiles}

\begin{document}

\chapter{Ebene hyperbolische Geometrie}
Knörrer: Geometrie Kapitel 3

\begin{goal}
    Konstruktion einer \emph{vollständigen} Fläche $H$ mit konstanter Krümmung $-1$, analog zur Ebene
    ($K \equiv 0$) und Sphäre ($K \equiv 1$).\\
    \emph{Vollständig}: Jede geodätische Kurve $\gamma : (a,b) \to H$ lässt sich geodätisch auf $\mathbb{R}$ erweitern.
\end{goal}

\begin{motivation}
    \emph{Gauss-Bonnet}
    $$\int _ \Sigma K \ dA = 2 \pi \chi (\Sigma)$$
    wobei $\Sigma$ eine kompakte, vollständige Fläche.
    Falls $\chi (\Sigma) < 0$ und die Krümmung $K$ konstant ist, dann muss $K$ negativ sein!

    \begin{theorem}[Klassifikation der Flächen]
        Sei $\Sigma$ eine topologische (glatte), kompakte, vollständige, orientierbare, zusammenhängende Fläche.
        Dann ist $\Sigma$ zu einer der Flächen $\Sigma _g$ homöomorph (diffeomorph): \\
        \begin{figure}[htb]
            \centering
            \def\svgwidth{20em}
            \input{figures/henkel.pdf_tex}
            \caption{$\Sigma _g$ mit $g$ Henkel}        
        \end{figure}

        Es gilt: $\chi (\Sigma _g)= 2-2g <0$ falls $g \ge 2$.

    \end{theorem}
\end{motivation}

\newpage
\section{Eine Riemannsche Metrik mit K=-1}
Naiver Ansatz zur Konstruktion einer Riemannschen Metrik auf $\mathbb{R}^2$ mit $K=-1$.
$$\langle \ , \ \rangle _p = h(p) \langle \ , \ \rangle_{\mathbb{R}^2}$$
wobei $h:\mathbb{R}^2 \to \mathbb{R}$ positiv und glatt ist. Für die Koeffizientenfunktionen $E, F, G$ gilt
also: 
\begin{itemize}
    \item $E(x,y) = \langle e_1, e_1 \rangle _{(x,y)} = h(x,y)$
    \item $F(x,y) = \langle e_1, e_2 \rangle _{(x,y)} = 0$
    \item $G(x,y) = \langle e_2, e_2 \rangle _{(x,y)} = h(x,y)$
\end{itemize}
\begin{terminology}
    Falls $E=G$ und $F=0$ gilt, dann heissen die Koordinaten \emph{konform} oder \emph{isotherm}.
\end{terminology}
Eine kleine Rechnung zeigt
$$K = - \frac{1}{2 h(x,y)} \Delta (\log (h(x,y)))$$ wobei $\Delta f = f_{xx} + f_{yy}$ der Laplaceoperator (siehe Serie 10).
\\
Nun führt $K=-1$ zu einer Differentialgleichung für $h$:
$$2 h(x,y)=\Delta (\log (h(x,y)))$$
Dies ist eine \emph{partielle Differentialgleichung}, welche schwierig zu lösen ist. Mit dem Lösungsansatz $h(x,y)=y^n$ finden wir eine Lösung
$h(x,y)=\frac{1}{y^2}$, welche allerdings nur auf der obenen Halbebene
$H= \{ z = x+ iy \in \mathbb{C} \ \vert \ y > 0 \}$ definiert ist.

\begin{definition}
    Die \emph{hyperbolische Ebene} ist die Menge $H= \{ z \in \mathbb{C} \ \vert \ \im (z) > 0 \}$ mit
    der Riemannschen Metrik
    $$\langle \ , \ \rangle _{x+iy} = \frac{1}{y^2} \langle \ , \ \rangle _{\mathbb{R}^2}$$
\end{definition}

\begin{remarks}
    \leavevmode
    \begin{enumerate}
        \item Die Translation $z \mapsto z + a$ mit $a \in \mathbb{R}$ ist eine Isometrie von $H$.
        Tatsächlich, schreibe
        \begin{align*}
            T : H & \to H \\ (x,y) & \mapsto (x+a,y)
        \end{align*}
        Für alle $p \in H$ gilt $(DT)_p = Id_{\mathbb{R}^2}$. Zu prüfen für alle $v,w \in \mathbb{R}^2$:
        $$\langle v,w \rangle _p \overset{?}{=} \langle (DT)_p(v),(DT)_p(w) \rangle _{T(p)} = \langle v,w \rangle _{T(p)}$$
        Stimmt, da $y(p)=y(T(p))$, und somit $\langle \ , \ \rangle _p = \langle \ , \ \rangle _{T(p)}$

        \item Die Streckung $z \mapsto \lambda z$ mit $\lambda >0$ ist eine Isometrie von $H$.
        Schreibe
        \begin{align*}
            S : H & \to H \\ (x,y) & \mapsto (\lambda x, \lambda y)
        \end{align*}
        Für alle $p \in H$ gilt $(DS)_p = \lambda Id_{\mathbb{R}^2}$. Zu prüfen für alle $v,w \in \mathbb{R}^2$:
        $$\langle v,w \rangle _p \overset{?}{=} \langle (DS)_p(v),(DS)_p(w) \rangle _{S(p)} = \lambda ^2 \langle v,w \rangle _{S(p)}$$
        Stimmt, da $y(S(p)) = \lambda y(p)$, also $\langle \ , \ \rangle _{S(p)} = \frac{1}{\lambda ^2} \langle \ ,\ \rangle _p$

        \item Die Inversion $z \mapsto -\frac{1}{z}$ ist eine Isometrie von $H$.
        Schreibe $$\varphi (z) = -\frac{1}{z} = - \frac{\bar{z}}{|z|^2}=\frac{-x+iy}{x^2+y^2} \in H$$ falls $z\in H$ d.h. $y>0$.
        Also $\varphi : H \to H$. Es gilt für alle $z \in H$ und $v \in \mathbb{C} = \mathbb{R}^2$:
        $$(D\varphi )_z (v)=\varphi ' (z) v = -\frac{1}{z^2}v$$
        Zu prüfen: $$\langle v, w \rangle _z \overset{?}{=} \langle -\frac{1}{z^2}v,-\frac{1}{z^2}w \rangle _{-\frac{1}{z}} = \frac{1}{|z|^4}\langle v,w\rangle _{-\frac{1}{z}}$$.
        Stimmt, da $y(-\frac{1}{z}) = \frac{1}{|z|^2} y(z) \implies \langle \ , \ \rangle _{-\frac{1}{z}} = |z|^4 \langle \ , \ \rangle _{z}$ 

    \end{enumerate}
\end{remarks}

\section{Möbiustransformationen}
\begin{recall}[aus der komplexen Analysis]
    Für $\begin{pmatrix}
        a & b \\ c & d
    \end{pmatrix} \in GL(\mathbb{C}^2)$, definieren wir die zugehörige \emph{Möbiustransformation} (nicht auf ganz $\mathbb{C}$ definiert).
    \begin{align*}
        (MT) \Phi : \mathbb{C} & \dashrightarrow \mathbb{C} \\
            z & \mapsto \frac{az+b}{cz+d}
    \end{align*}

\end{recall}

\begin{examples}
    \leavevmode
    \begin{enumerate}
        \item $A = \begin{pmatrix}
            1 & b \\ 0 & 1
        \end{pmatrix} b \in \mathbb{C} \implies \Phi _A (z)=z+b$

        \item $A = \begin{pmatrix}
            a & 0 \\ 0 & a^{-1}
        \end{pmatrix} a \in \mathbb{C}\setminus \{0\} \implies  \Phi _A (z)=a^2z$.
        Für $a = \sqrt{\lambda}: \lambda z \ (\lambda > 0)$

        \item $A = \begin{pmatrix}
            0 & 1 \\ -1 & 0
        \end{pmatrix} \implies \Phi _A (z)= -\frac{1}{z}$
        Insbesondere, für $a = \sqrt{\lambda} (\lambda > 0): \lambda z$
    \end{enumerate}
\end{examples}

\begin{remark}
    Die obigen Isometrien 1-3 sind vom Typ $\Phi _A$ mit $A \in SL(\mathbb{R}^3)$.
    (Determinante $1$)
\end{remark}

\subsection*{Projektive Interpretation von Möbiustransformation}
Sei $A \in GL(\mathbb{C}^2)$. Dann erhalten wir eine lineare Abbildung
$A : \mathbb{C}^2 \to \mathbb{C}^2$. Insbesondere bildet $A$ Geraden durch 0 auf
Geraden durch 0 ab (1).
\begin{definition}
    Die \emph{projektive Gerade} $\mathbb{P}(\mathbb{C}^2) = \mathbb{P}^1\mathbb{C}$ ist die Menge
    aller komplexen Geraden durch 0 in $\mathbb{C}^2$. Konkret:
    Die Menge der Äquivalenzklassen bezüglich folgender Äquivalenzrelation auf $\mathbb{C}^2 \setminus \{0\}$:
    $$v \sim w \iff \exists \lambda \in \mathbb{C}, \lambda \not = 0 \text{ mit }w = \lambda v$$
\end{definition}

Dann ist 
$\mathbb{P}(\mathbb{C}^2) = \big (\mathbb{C}^2 \setminus \{0\} \big ) / \sim$.
Sei nun $\begin{pmatrix}
    a \\ b
\end{pmatrix} \in \mathbb{C}^2 \setminus \{0\}$
\begin{itemize}
    \item Falls $b \not = 0$, dann gilt 
    $v=\begin{pmatrix}
        a \\b
    \end{pmatrix} \sim \begin{pmatrix}
    \frac{a}{b} \\1
    \end{pmatrix} = \begin{pmatrix}
        z \\1
        \end{pmatrix}
    z \in \mathbb{C}$.

    \item Falls $b=0$, dann gilt
    $v=\begin{pmatrix}
        a \\b
    \end{pmatrix} \underset{a \not = 0}{\sim} \begin{pmatrix}
    1 \\ 0
    \end{pmatrix} = \infty$.

\end{itemize}
Daraus folgern wir, dass $\mathbb{P}(\mathbb{C}^2) = \mathbb{C} \cup \{ \infty \}$.
Aus (1) folgt: Die Abbildung $A : \mathbb{C}^2 \to \mathbb{C}^2$ induziert eine Abbildung
\begin{align*}
    \Phi _A : \mathbb{P}(\mathbb{C}^2) & \to \mathbb{P}(\mathbb{C}^2) \\
    [v] & \mapsto [Av]
\end{align*}
Interpretation via $\mathbb{P}(\mathbb{C}^2) = \mathbb{C} \cup  \{ \infty \}$.

\begin{itemize}
    \item $v = \begin{pmatrix}
        z \\ 1
    \end{pmatrix} \implies Av = \begin{pmatrix}a&b\\c&d\end{pmatrix}\begin{pmatrix}z \\ 1\end{pmatrix}=\begin{pmatrix}az+b\\cz+d\end{pmatrix}
    \sim \begin{pmatrix}\frac{az+b}{cz+d} \\ 1\end{pmatrix}\text{ bzw. } cz+d=0 \sim \begin{pmatrix}1\\0\end{pmatrix}$\nolinebreak

    \item $v = \begin{pmatrix}1 \\ 0 \end{pmatrix} \implies Av = \begin{pmatrix}a&b\\c&d\end{pmatrix}\begin{pmatrix}1 \\ 0\end{pmatrix}=\begin{pmatrix}a \\ c \end{pmatrix}
    \sim \begin{cases}
        \begin{pmatrix}\frac{a}{c}\\1 \end{pmatrix}\text{ falls } c \not = 0 \\
        \begin{pmatrix}1\\0 \end{pmatrix} \text{ falls } c = 0
            
    \end{cases}$
\end{itemize}

Notation für $\Phi _A : \Phi _A(z)= \dfrac{az+b}{cz+d}$
``geeignet interpretiert''.
Aus dieser Definition folgt auch dass $\Phi _{AB} = \Phi _A \circ \Phi _B$. Diese Tatsache ist mit der anderen Definition
mühsam zu beweisen.

\begin{lemma}
    Sei $A =\begin{pmatrix}
        a&b\\c&d
    \end{pmatrix}\in SL(\mathbb{R}^2)$
    Dann erhält die Möbiustransformation $\varphi _A$ die obere Halbebene
    $H=\{ z\in \mathbb{C} \ | \ \im (z)>0\}$.
\end{lemma}

\begin{proof}
    Sei $z \in H$, d.h. $\im (z)= \frac{1}{2i} (z- \bar{z}) >0$.
    Berechne
    \begin{align*}
        \im(\varphi _A (z)) &= \frac{1}{2i}\left( \frac{az+b}{cz+b}-\frac{a\bar{z}+b}{c\bar{z}+d} \right) \\
        &= \frac{1}{2i} \frac{(az+b)(c\bar{z}+d)+(a\bar{z}+b)(cz+d)}{|cz+d|^2} \\
        &= \frac{1}{2i} \frac{(ad-bc)(z-\bar{z})}{|cz+d|^2} \\
        & \overset{det A=1}{=} \frac{\im (z)}{(cz+d)^2} >0
    \end{align*}
\end{proof}
\begin{remark}
    Es gilt sogar $\varphi _A(H) = H$
    Tatsächlich gilt 
    \begin{align*}
        H=\varphi _{\begin{pmatrix}
            1 & 0 \\ 0  & 1
        \end{pmatrix}}(H) &= \varphi _{A \circ A^{-1}}(H) \\
        &\overset{\varphi_{AB}=\varphi_A \circ \varphi_B}{=} \varphi _A \circ \varphi _{A^{-1}}(H)=\varphi_A (\varphi _{A^{-1}}(H))\subset \varphi _A (H)
    \end{align*} $\implies \varphi _A (H)=H$. Daraus folgt, dass Möbiustransformationen eine Gruppe bilden.
    
\end{remark}

\begin{lemma}
    Jede Möbiustransformation $\varphi _A : H \to H$ mit $A \in SL(\mathbb{R}^2)$ ist eine endliche
    Komposition von Möbiustransformationen der Form
    \begin{enumerate}
        \item $z \mapsto z + b \qquad (b\in \mathbb{R})$ horizontale Translation
        \item $z \mapsto \lambda z \qquad (\lambda >0)$ Streckung
        \item $z \mapsto -\frac{1}{z} $ Inversion
    \end{enumerate}
\end{lemma}

\begin{proof}
    \begin{align*}
        \frac{az +b}{cz+d} =\frac{a}{c} \frac{cz+\frac{c}{a}b}{cz +d}=\frac{a}{c}\frac{cz+d+(\frac{c}{a}b-d)}{cz+d}
        = \alpha + \frac{\beta}{cz+d}
    \end{align*} für geeignete $\alpha$ und $\beta$.
    Details siehe Serie 11.
\end{proof}
\begin{corollary}
    Alle Möbiustransformationen der Form $\varphi _A : H \to H$ mit $A \in SL(\mathbb{R}^2)$
    sind Isometrien bezüglich der Riemannschen Metrik $\frac{1}{y^2}\langle \ , \ \rangle _{\mathbb{R}^2}$.
\end{corollary}
\begin{proof}
    Möbiustransformationen des Typs 1-3 sind Isometrien, siehe oben
\end{proof}
\begin{lemma}
    Die Kurve $\begin{aligned}[t]
        \gamma : \mathbb{R} & \to H \\
        t & \mapsto i e^t
    \end{aligned}$
\end{lemma}

\begin{proof}
    Wir bemerken zuerst, dass
    \begin{align*}
        || \dot{\gamma}(t)||_H &= \sqrt{\langle \dot{\gamma}(t), \dot{\gamma}(t)\rangle _H} \\
        &\overset{y(\gamma(t))=e^t}{=} \sqrt{\frac{1}{(e^t)^2} \underbrace{\langle i e^t, i e ^t \rangle _{\mathbb{R}^2}}_{\langle e^t, e^t \rangle = (e^t)^2}}=1
    \end{align*}
    $\implies \gamma :\mathbb{R}\to H $ ist nach Bogenlänge parametrisiert (bzgl. hyperbolischer Metrik).
    Sei nun $\delta : \mathbb{R} \to H$ die eindeutige geodätische Kurve mit
    $\delta (0) =i$ und $\dot{\delta}(0)=i$.
    Betrache die folgende Isometrie von $H$ (Spiegelung an $i \mathbb{R})$
    \begin{align*}
        \sigma : H & \to H \\
        x +iy & \mapsto -x + iy
    \end{align*}
    Nun ist $\sigma \circ \delta :\mathbb{R}\to H$ auch geodätisch mit $\sigma \circ \delta(0)=i$
    und auch $\frac{d}{dt}(\sigma \circ \delta)(0)=i$.
    Aus der Eindeutigkeit der Geodäten zu Anfangsbedingungen folgt also $\delta = \sigma \circ \delta$,
    also $\delta (\mathbb{R}) \subset i \mathbb{R}$.
    Da $\gamma$ und $\delta $ nach Bogenlänge parametrisiert ($\delta$ ist geodätisch mit $||\dot{\delta}(0)||_H =1$!) sind, folgt $\gamma = \delta$. 
\end{proof}

\begin{proposition}
    Die Geodäten in $H$ sind genau die Halbgeraden und Halbkreise, welche senkrecht auf ``$\mathbb{R} \cup \{\infty \} = \partial H$''.
    stehen. 
\end{proposition}

\begin{figure}[htb]
    \centering
    \def\svgwidth{20em}
    \input{figures/proposition_circles.pdf_tex}
    \caption{test}        
\end{figure}

\begin{proof}
    Wir haben schon eine Geodäte gefunden:
    $i \mathbb{R}_{>0} $, das Bild der Kurve $\gamma (t) = ie^t$.
    Schreibe $h = \bild (\gamma) \subset H$. Nun ist für jede Isometrie
    $\varphi : H \to H, \varphi (H) \subset H$ auch eine Geodäte. Insbesondere
    können wir auf $h$ iteriert Abbildung der Form
    \begin{enumerate}
        \item $z \mapsto z + b \qquad b \in \mathbb{R}$
        \item $z \mapsto \lambda z \qquad \lambda > 0$
        \item $z \mapsto -\frac{1}{z}$
    \end{enumerate}
    Daraus folgt, dass alle Halbgeraden auf $\mathbb{R}$ Geodäten sind.
    Betrachte die spezielle Isometrie $\varphi (z)=-\frac{2}{z+1}$
    \begin{claim}
        $\varphi (h)$ ist ein Halbkreis in $H$ mit Zentrum $-1$ und Radius $1$
    \end{claim}
    \begin{proof}
        Sei $i y \in h$. Berechne
        \begin{align*}
            |\varphi(iy)+1| &= \left| -\frac{2}{iy+1} + \frac{iy+1}{iy+1} \right| \\
            &= \left| \frac{iy -1}{iy+1} \right| = 1
        \end{align*}
    \end{proof}
    Unter Anwendung von horizontalen Transformationen und Streckungen erhalten wir aus $\varphi(h)$
    alle Halbkreise Senkrecht auf $\mathbb{R}$.
    %\ctikzfig{Halbkreis_proof}

    \begin{question}
        Wieso existieren keine weiteren Geodäten?
    \end{question}
    Zu jeden $z \in H$ und jedem Einheitsvektor $v$ existiert genau eine geodätische Kurve $\gamma : \mathbb{R} \to H$
    mit $\gamma (0) = z$ und $\dot{\gamma} (0) = v$. Das Bild von $\gamma$ muss also der Halbkreis
    oder die Halbgerade durch $z$ mit Tangente $v$ sein!
    %\ctikzfig{kleine_zeichnung_am_rande}
\end{proof}

\begin{remark}
    Für alle $z, w \not = z \in H$ existiert eine Geodäte $g < H$ mit $z,w \in g$
    %\ctikzfig{bild_in_bemerkung_1}.
    Hingegen existiert zu $g \subset H$ und $z \not \in g$ unendlich viele Geodäten $h \in H$ mit $z \in h$ und $h \cap g = \emptyset$.
    %\ctikzfig{bild_in_bemerkung_2}
    Wir bemerken, dass das \emph{Parallelaxiom} in der hyperbolischen Ebene \emph{nicht erfüllt} ist.
\end{remark}

\section{Die Isometriegruppe von H}
\begin{lemma}
    Sei $\varphi : H \to H$ eine orientierungserhaltende Isometrie,
    d.h. für alle $z \in H$ gilt $\det \left ((D\varphi)_z > 0 \right )$. Dann ist $\varphi$ durch
    $\varphi (i)$ und $(D\varphi)(i)$ eindeutig bestimmt.
\end{lemma}
\begin{proof}
    Geometrisch, unter Benutzung der Tatsache, dass Isometrien winkelerhaltend sind. Wir bemerken zuerst,
    dass die Einschränkung von $\varphi $ auf $i \mathbb{R}_{>0}$ durch $\varphi (i)$ und
    $(D\varphi)_i(i)$ bestimmt ist:
    $\delta (1)=\varphi (ie^1)$ ist die eindeutige Geodäte mit $\delta (0) = \varphi (i)$ und
    $\dot{\delta}(0) = (D\varphi)_i(i)$! Insbesondere kennen wir auch $\varphi (2i)\in H$.
    Aus $\varphi (i)$ und $\varphi(2i)$ können wir für \emph{alle} $z \in H \varphi (z)$ bestimmen:
    %\ctikzfig{wirre_zeichnung_von_lemma} 
\end{proof}

\begin{definition}
    $\Iso ^+ (H) = \{ \varphi : H \to H \ | \ \varphi \text{ ist eine orientierungserhaltende Isometrie}\}$ 
\end{definition}

Wir wissen bereits:



\end{document}
