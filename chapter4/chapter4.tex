\documentclass[../main.tex]{subfiles}

\begin{document}

\chapter{Ebene hyperbolische Geometrie}
Knörrer: Geometrie Kapitel 3

\begin{goal}
    Konstruktion einer \emph{vollständigen} Fläche $H$ mit konstanter Krümmung $-1$, analog zur Ebene
    ($K \equiv 0$) und Sphäre ($K \equiv 1$).\\
    \emph{Vollständig}: Jede geodätische Kurve $\gamma : (a,b) \to H$ lässt sich geodätisch auf $\mathbb{R}$ erweitern.
\end{goal}

\begin{motivation}
    \emph{Gauss-Bonnet}
    $$\int _ \Sigma K \ dA = 2 \pi \chi (\Sigma)$$
    wobei $\Sigma$ eine kompakte, vollständige Fläche.
    Falls $\chi (\Sigma) < 0$ und die Krümmung $K$ konstant ist, dann muss $K$ negativ sein!

    \begin{theorem}[Klassifikation der Flächen]
        Sei $\Sigma$ eine topologische (glatte), kompakte, vollständige, orientierbare, zusammenhängende Fläche.
        Dann ist $\Sigma$ zu einer der Flächen $\Sigma _g$ homöomorph (diffeomorph): \\
        \begin{figure}[htb]
            \centering
            \def\svgwidth{20em}
            \input{figures/henkel.pdf_tex}
            \caption{$\Sigma _g$ mit $g$ Henkel}        
        \end{figure}

        Es gilt: $\chi (\Sigma _g)= 2-2g <0$ falls $g \ge 2$.

    \end{theorem}
\end{motivation}

\newpage
\section{Eine Riemannsche Metrik mit K=-1}
Naiver Ansatz zur Konstruktion einer Riemannschen Metrik auf $\mathbb{R}^2$ mit $K=-1$.
$$\langle \ , \ \rangle _p = h(p) \langle \ , \ \rangle_{\mathbb{R}^2}$$
wobei $h:\mathbb{R}^2 \to \mathbb{R}$ positiv und glatt ist. Für die Koeffizientenfunktionen $E, F, G$ gilt
also: 
\begin{itemize}
    \item $E(x,y) = \langle e_1, e_1 \rangle _{(x,y)} = h(x,y)$
    \item $F(x,y) = \langle e_1, e_2 \rangle _{(x,y)} = 0$
    \item $G(x,y) = \langle e_2, e_2 \rangle _{(x,y)} = h(x,y)$
\end{itemize}
\begin{terminology}
    Falls $E=G$ und $F=0$ gilt, dann heissen die Koordinaten \emph{konform} oder \emph{isotherm}.
\end{terminology}
Eine kleine Rechnung zeigt
$$K = - \frac{1}{2 h(x,y)} \Delta (\log (h(x,y)))$$ wobei $\Delta f = f_{xx} + f_{yy}$ der Laplaceoperator (siehe Serie 10).
\\
Nun führt $K=-1$ zu einer Differentialgleichung für $h$:
$$2 h(x,y)=\Delta (\log (h(x,y)))$$
Dies ist eine \emph{partielle Differentialgleichung}, welche schwierig zu lösen ist. Mit dem Lösungsansatz $h(x,y)=y^n$ finden wir eine Lösung
$h(x,y)=\frac{1}{y^2}$, welche allerdings nur auf der obenen Halbebene
$H= \{ z = x+ iy \in \mathbb{C} \ \vert \ y > 0 \}$ definiert ist.

\begin{definition}
    Die \emph{hyperbolische Ebene} ist die Menge $H= \{ z \in \mathbb{C} \ \vert \ \Im (z) > 0 \}$ mit
    der Riemannschen Metrik
    $$\langle \ , \ \rangle _{x+iy} = \frac{1}{y^2} \langle \ , \ \rangle _{\mathbb{R}^2}$$
\end{definition}

\begin{remarks}
    \leavevmode
    \begin{enumerate}
        \item Die Translation $z \mapsto z + a$ mit $a \in \mathbb{R}$ ist eine Isometrie von $H$.
        Tatsächlich, schreibe
        \begin{align*}
            T : H & \to H \\ (x,y) & \mapsto (x+a,y)
        \end{align*}
        Für alle $p \in H$ gilt $(DT)_p = Id_{\mathbb{R}^2}$. Zu prüfen für alle $v,w \in \mathbb{R}^2$:
        $$\langle v,w \rangle _p \overset{?}{=} \langle (DT)_p(v),(DT)_p(w) \rangle _{T(p)} = \langle v,w \rangle _{T(p)}$$
        Stimmt, da $y(p)=y(T(p))$, und somit $\langle \ , \ \rangle _p = \langle \ , \ \rangle _{T(p)}$
    \end{enumerate}
\end{remarks}


\end{document}
