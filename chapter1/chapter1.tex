\documentclass[../main.tex]{subfiles}

\begin{document}

\chapter{Untermannigfaltigkeiten und Flächen}
Zentral für das Verständnisses dieser Vorlesung sind Untermannigfaltigkeiten (UMF). Der Prototyp einer Untermannigfaltigkeit hat immer die Form
\begin{align*}
    \mathbb{R}^k \times \{0\} \subset \mathbb{R}^n \quad \text{wobei } k<n
\end{align*}
Eine UMF sollte bis auf lokale Diffeomorphismen ``so aussehen''

\ctikzfig{fig1}

\begin{definition}
Seien $U,V \subset \mathbb{R}^n$ offen. Eine Abbildung $\varphi : U \to V$ heisst Diffeomorphismus falls $\varphi$ bijektiv, und sowohl $\varphi$ und $\varphi ^{-1}$ unendlich oft differenzierbar. Schreibe auch $\varphi\in(C^\infty)$ bzw. $\varphi$ glatt.
\end{definition}
\begin{remark}
In der Literatur wird manchmal auch nur $C^1$, also stetig differenzierbar gefordert.
\end{remark}

\begin{examples}
\leavevmode
\begin{enumerate}
    \item $\begin{aligned}[t] 
        \varphi \colon \mathbb{R} & \to \mathbb{R}_{>0} \text{ ist ein Diffeomorphismus mit Umkehrung} \log \colon \mathbb{R}_{>0} \to \mathbb{R} \\ x & \mapsto e^x
    \end{aligned}$
    
    \item $\begin{aligned}[t] 
        \varphi \colon \left(-\frac{\pi}{2},\frac{\pi}{2} \right) & \to \mathbb{R} \text{ ist ein Diffeomorphismus mit } \varphi^{-1} = \arctan \\ x & \mapsto \tan(x)
    \end{aligned}$
    
    \item $\begin{aligned}[t] 
        \varphi \colon \mathbb{R} & \to \mathbb{R} \text{ ist \emph{kein} Diffeomorphismus, } \varphi^{-1}(x) = \sqrt[3]{x} \text{ bei } x=0 \text{ nicht diff'bar} \\ x & \mapsto x^3
    \end{aligned}$
\end{enumerate}
\end{examples}

\begin{recall}[Umkehrsatz]
Sei $f \colon \mathbb{R}^n \to \mathbb{R}^n$ in $C^1$ (also stetig diff'bar) und $p \in \mathbb{R}^n$ mit $\det(Df)_p \not = 0$. Dann existieren $U,V \in \mathbb{R}^n$ offen mit $p \in U$ und $V=f(U)$, so dass die Einschränkung $f\vert_U \colon U \to V$ ein Diffeomorphismus (im $C^1$-Sinn ist). ``f hat bei $p$ eine lokale Umkehrung in $C^1$.''
\end{recall}

\section{Untermannigfaltigkeit}

\begin{definition}
Eine abgeschlossene Teilmenge $M \subset \mathbb{R}^n$ heisst Untermannigfaltigkeit der Dimension $k$ falls
$\forall p \in M$ zwei offene Mengen $U,V \subset \mathbb{R}^n$ mit $p \in U$ und $0 \in V$ existieren, sowie ein $C^1$-Diffeomorphismus $\varphi \colon U \to V$ mit $\varphi \left(U \cap M\right) = \left(\mathbb{R}^k \times \{0\}\right) \cap V$ und $\varphi(p)=0$
\end{definition}

\ctikzfig{umf}

\begin{example}
\begin{minipage}[t]{30em}
Die $ \left( \text{x-Achse } \cup \text{ y-Achse}\right) \setminus \{0\} = M$ ist zwar lokal diffeomorph zu $\mathbb{R} \times \{0\} \subset \mathbb{R}^2$, ist aber nicht abgeschlossen. Also \emph{keine} UMF in $\mathbb{R}^2$! 
\end{minipage}
\begin{minipage}{1em}
\tikzfig{bsp_non_umf}
\end{minipage}

\end{example}

\begin{question}
Wie konstruieren wir nicht triviale Beispiele von Untermannigfaltigkeiten?
\end{question}

\begin{definition}
Sei $f \colon \mathbb{R}^n \to \mathbb{R}^m \ C^1$ mit $m<n$.
Ein Punkt $p$ heisst \emph{regulär} (für f), falls das Differential $\left(Df\right)_p \colon \mathbb{R}^n
\to \mathbb{R}^m$ surjektiv ist. Ein Wert $w \in \mathbb{R}^m$ heisst \emph{regulär}, falls alle $p \in f^{-1}(w)$ regulär sind. Nicht reguläre Punkte/Werte heissen \emph{kritisch}.
\end{definition}

\begin{remark}
Falls $w \not \in \bild (f)$, dann ist $w$ auch regulär.
\end{remark}

\noindent\fbox{\begin{minipage}{30em}
Im Spezialfall $k=2$ und $n=3$ heisst $M \subset \mathbb{R}^3$ eine \emph{reguläre Fläche}.
\end{minipage}}

\begin{theorem}
Sei $f \colon \mathbb{R}^n \to \mathbb{R}^m \ C^1$ mit $m\leq n$ und $w \in \mathbb{R}^m$ ein regulärer Wert.
Dann ist das Urbild $f^{-1}(w) = \{ p \in \mathbb{R}^n \ \vert \ f(p) = w \} \subset \mathbb{R}^n$ eine Untermannigfaltigkeit der Dimension $n-m$.
\end{theorem}

\begin{examples}
\leavevmode
\begin{enumerate}
    \item $\begin{aligned}[t]
        f \colon  \mathbb{R}^3 & \to \mathbb{R} \\
        p  & \mapsto \langle p, e_3 \rangle
    \end{aligned}$
    Berechne $\forall p \in \mathbb{R}^3$ das Differential
    $\begin{aligned}[t]\left(Df\right)_p \colon \mathbb{R}^3 & \to \mathbb{R} \\
    h & \mapsto \langle h,e_3 \rangle
    \end{aligned}$
    
    \begin{recall}[Dreigliedentwicklung]
    \begin{align*}
    f(p+h) &= f(p) + \left(Df\right)_p(h) + \left(Rf\right)_p(h) \\
    \langle p+h, e_3 \rangle &= \langle p, e_3 \rangle + \langle h, e_3 \rangle + 0
    \end{align*}
    \end{recall}
    Insbesondere ist $\forall p\in \mathbb{R}^3 \left(Df\right)_p \colon \mathbb{R}^3 \to \mathbb{R}$ surjektiv, da der Gradient $\\ \left(\nabla f \right)_p = e_3 \not = 0$ \quad \emph{Gradient}: Für $f \colon \mathbb{R}^n \to \mathbb{R}$ gilt $\left(Df\right)_p(h) = \langle \left(\nabla f\right)_p, h \rangle$. \\
    Wir folgern, dass für alle $w \in \mathbb{R}$ die Menge $f^{-1}(w)=\mathbb{R}^2 \times \{0\} \subset \mathbb{R}^3$ eine Untermannigfaltigkeit der Dimension $3-1=2$
    
    \item $\begin{aligned}[t]
        f \colon  \mathbb{R}^3 & \to \mathbb{R} \\
        p  & \mapsto \langle p, p \rangle = ||p||^2 _2
    \end{aligned}$
    \quad (In Koordinaten $f(x,y,z)=x^2 + y^2 + z^2$) \\ \\ Berechne $\forall p \in \mathbb{R}^3$
    \begin{align*}
        \left(Df\right)_p \colon \mathbb{R}^3 & \to \mathbb{R} \\
        h & \mapsto 2\langle p, h \rangle
    \end{align*}
    \begin{align*}
        f(p+h) = \langle p+h, p+h \rangle = \underbrace{\langle p,p \rangle}_\textrm{$f(p)$} +
        \underbrace{2\langle p,h \rangle}_\textrm{$(Df)_p(h)$} +
        \underbrace{\langle h,h \rangle}_\textrm{$(Rf)_p(h)$}
    \end{align*}
    $\implies (\nabla f)_p = 2p$ \quad Also ist $(Df)_p \colon \mathbb{R}^3 \to \mathbb{R}$ surjektiv $\iff p \not =0$\\
    Wir folgern, dass für alle $w \not = 0$ die Menge $f^{-1}(w) = \{p \in \mathbb{R}^3 \ \vert \ ||p||^2 _2 = w \}$ eine Untermannigfaltigkeit der Dimension $3-1 = 2$ ist.\\
    
    \begin{minipage}{2.5em} \emph{$w < 0$} \begin{align*} \emptyset \end{align*}
    \end{minipage}
    \qquad
    \begin{minipage}{5em} \emph{$w = 0$} \begin{align*} \Dot{} \quad 0 \end{align*}
    \end{minipage}
    \begin{minipage}{5em} \emph{$w > 0$} \begin{align*} \end{align*}
    \end{minipage}
    \tikzfig{sphere_umf}
    
    \item
    $f \colon \mathbb{R}^3 \to \mathbb{R}, \ f(x,y,z)=x^2+y^2-z^2$.
    Berechne $(\nabla f)_p = (2x, 2y, -2z), p = (x,y,z)$
    Also $(Df)_p \colon \mathbb{R}^3 \to \mathbb{R}$ surjektiv $\iff p \not = 0$.
    \emph{Folgerung} $w \not = 0 \implies f^{-1} \subset \mathbb{R}^3$ UMF der Dimension $2$.
    \ctikzfig{third_example_umf}
\end{enumerate}
\end{examples}

\begin{proof}[Beweis von Theorem 1]
Sei $p \in \mathbb{R}^n$ mit $f(p)=w$. Dann ist $(Df)_p \colon \mathbb{R}^n \to \mathbb{R}^m$ surjektiv.
Nach dem Satz über implizite Funktionen existieren \emph{offene} Mengen:
\begin{itemize}
    \item $A \subset K = \ker\left((Df)_p\right)$
    \item $B \subset \mathbb{R}^m$
    \item $Y \subset \mathbb{R}^n$
\end{itemize}
mit $0\in A, \ w\in B$ und $p \in V$ sowie ein $C^1$-Diffeomorphismus $g\colon A \times B \to V$ mit
\begin{enumerate}[i)]
    \item $f(g(k,y))=y \ \forall k \in A, \ y\in B$
    \item $g(k,y) - (p+k) \in K^{\perp}$
\end{enumerate}
Die Niveaumenge $f^{-1}$ ist lokal gleich dem Graphen einer Funktion der Form $k \mapsto g(k,y)$.
Insbesondere gilt: $f^{-1} \cap V = g(A\times \{w\})$. Wir folgern, dass $f^{-1}(w)$ eine UMF (via $\varphi = g^{-1}$) der Dimension $k=n-m$ ist. Da $\dim K = \dim (\mathbb{R}^n)- \dim(\bild(Df)_p)$. \newline \newline
\emph{Bild}:
\ctikzfig{proof_theorem1}
\end{proof}

\newpage

\section{Lokale Parametrisierung von Untermannigfaltigkeiten}
\begin{definition}
Eine $C^1$-Abilldung $f : \mathbb{R}^k \to \mathbb{R}^n$ heisst
\emph{lokale Einbettung} beim Punkt $p \in \mathbb{R}^k$ falls $(Df)_p : \mathbb{R}^k \to \mathbb{R}^n$ \emph{injektiv} ist. $(k \leq n$) 
\end{definition}

\begin{theorem}
Sei $f : \mathbb{R}^k \to \mathbb{R}^n$ eine lokale Einbettung bei $p \in \mathbb{R}^k$. Dann existiert eine offene Menge
$W \subset \mathbb{R}^k$ mit $p \in W$, sodass $f(W) \subset \mathbb{R}^n $ eine \emph{offene Teilmenge} einer Untermannigfaltigkeit $M \subset \mathbb{R}^n$ ist. \newline
In diesem Kontext heisst $f$ eine \emph{lokale Parametrisierung} von $M$ bei $f(p) \in M$.
\end{theorem}

\begin{examples}
\leavevmode
\begin{enumerate}
    \item $M = \mathbb{R}^2 \times \{0\} \subset \mathbb{R}^3$. Dann ist \begin{align*}
        \varphi : \mathbb{R}^2 & \to \mathbb{R}^3 \\
        (u,v) & \mapsto (u,v,0)
    \end{align*}
    eine (globale) Parametrisierung von $M$. Tatsächlich ist $\varphi$ linear, mit Abbildungsmatrix 
    $A =  \left(\begin{smallmatrix} 1 & 0 \\ 0 & 1 \\ 0 & 0
    \end{smallmatrix} \right)$.
    Die Jakobimatrix $(J\varphi )_p$ stimmt in jedem Punkt mit $A$ überein $\implies (D\varphi )_p : \mathbb{R}^2 \to \mathbb{R}^3$ injektiv.
    
    \item
    $M = S^2 = \{ p \in \mathbb{R}^3 \vert \ ||p||_2 = 1 \} \subset \mathbb{R}^3$ (reguläre Fläche, siehe oben).
    Definiere \begin{align*}
        \varphi \colon D^2 & \to S^2 \\
        (u,v) &\mapsto (u,v, \sqrt{1-u^2-v^2})
    \end{align*}
    Bereche $(J \varphi)_{(0,0,0)} = \left(\begin{smallmatrix} 1 & 0 \\ 0 & 1 \\ 0 & 0
    \end{smallmatrix} \right)$ Also ist $\varphi$ eine lokale Parametrisierung von $S^2$ beim Nordpol $N=(0,0,1)=\varphi(0,0)$.
    
    
    \item
    Seien $a,b \in \mathbb{R} $ mit $0 < a < b$. Definiere den Rotationstorus \\
    \begin{minipage}[b]{30em}
        
        \begin{align*}
        \varphi \colon \mathbb{R}^2 & \to T \subset \mathbb{R}^3 \\
        (u,v) & \mapsto \begin{pmatrix}(b+a \cos u)\cos v
        \\(b+a \cos u)\sin v
        \\ a \sin u\end{pmatrix}
    \end{align*}
    In jedem Punkt $(u,v) \in \mathbb{R}^2$ ist $\varphi$ eine lokale Parametrisierung von $T$. Berechne dazu
    \end{minipage}
    \begin{minipage}[b]{10em}\tikzfig{torus}
    \end{minipage}
    
    
    $(J \varphi)_{u,v} = \begin{pmatrix}
    -a \sin u \cos v & -(b+ a \cos u) \sin v \\
    -a \sin u \sin v & (b+ a \cos u) \cos v \\
    a \cos u & 0
    \end{pmatrix}$. \\
    Es gilt $\forall (u,v) \in \mathbb{R}^2 \colon \Rang (J\varphi )_{(0,0)} = 2 \implies (D\varphi )_{(u,v)} : \mathbb{R}^2 \to \mathbb{R}^3$ injektiv.
\end{enumerate}
\end{examples}
\begin{question}
Besitzt jede Untermannigfaltigkeit $M \subset \mathbb{R}^n$ bei jedem Punkt eine lokale Parametrisierung?
\end{question}

\begin{proposition}
Jede Untermannigfaltigkeit $M \subset \mathbb{R}^n$ besitzt bei allen Punkten $p \in M$ eine lokale Parametrisierung.
\end{proposition}
\begin{proof}
Da $M \subset \mathbb{R}^n$ eine UMF ist, existieren für alle $p \in M$ offene Mengen $U,V \subset \mathbb{R}^n$ mit $p \in U, \ 0 \in V$ sowie $g : U \to V$ ein Diffeomorphismus mit $g(p)=0$ und \\
$g(U \cap M )=(\mathbb{R}^k \times \{0\} ) \cap V$.
Definiere nun \begin{align*}
    \varphi : g^{-1} \vert_{(\mathbb{R}^k \times \{0\})\cap V} \colon (\mathbb{R}^k \times \{0\})\cap V
    & \to M \cap U \\
    q & \mapsto g^{-1}(q)
\end{align*}
Es gilt: $\varphi (0) = p$ und $(D\varphi )_0 = \left(Dg^{-1}\right)_0 \vert_{ \mathbb{R}^k \times \{0\}}
 = (Dg)^{-1}_{p} \vert_{\mathbb{R}^k \times \{0\}}$. Nach Konstruktion \\(g ist ein Diffeomorphismus) ist
 $(Dg)_p : \mathbb{R}^n \to \mathbb{R}^n$ ist ein Isomorphismus, ebenso \\$(Dg)_p ^{-1} : \mathbb{R}^n \to \mathbb{R}^n \implies (Dg)^{-1}_{p} \vert_{\mathbb{R}^k \times \{0\}} \text{ist injektiv} \implies
 (D\varphi )_0 : \mathbb{R}^k \times \{0\} \to \mathbb{R}^n$ injektiv. Also ist $\varphi$ eine lokale Parametrisierung bei $p \in M$.
\end{proof}
\begin{proof}[Beweis von Theorem 2]
Zur Vereinfachung nehmen wir $p=0$ und $f(p)=0$ an.
Setze $B = \bild(Df)_p < \mathbb{R}^n$, ein Untervektorraum der Dimension $k$ (da $(Df)_p$ injektiv.
Weiterhin $S = B^{\perp} < \mathbb{R}^n$, sodass gilt $\mathbb{R}^n = B \oplus S, \ \dim (S) = n-k$.
Definiere \begin{align*}
    F : \mathbb{R}^k \times S & \to \mathbb{R}^n \\
    (q,s) & \mapsto f(q) + s
\end{align*}
$F$ ist stetig differenzierbar. Dreigliedentwicklung im Punkt $(p,0)=\mathbb{R}^k \times S \colon \\$
Sei $h_1 \in \mathbb{R}^k, h_2 \in S$. Berechne:
\begin{align*}
    F(p+h_1, 0 + h_2) = f(p+h_1) + h_2 = f(p) + (Df)_p(h_1)+ (Rf)_p(h_1) + h_2 \\
    \implies (DF)_{(p,0)}\left((h_1,h_2)\right) = (Df)_p(h_1) + h_2
\end{align*}
\begin{claim}
$(DF)_{(p,0)} : \mathbb{R}^k \times S \to \mathbb{R}^n$ ist ein Isomorphismus
\end{claim}
\begin{proof}
Es reicht zu zeigen: $(DF)_{(p,0)}$ ist injektiv (da $k + (n-k) = n)$.\\
Sei $(h_1,h_2) \in \ker(DF)_{(p,0)}$
\begin{align*}
    %(Df)_p(h_1) + h_2 =0
    \underbrace{(Df)_p(h_1)}_\textrm{$\in B$} + \underbrace{h_2}_\textrm{$\in S$} = 0 \underbrace{\implies}_\textrm{$B \cap S = \{0\}$} (Df)_p(h_1) = 0 \text{ und } h_2 = 0
    \underbrace{\implies}_\textrm{$(Df)_p$ injektiv} h_1 = 0
\end{align*}
Da also $\ker(DF)_{(p,0)} = 0$, ist $(DF)_{(p,0)}$ injektiv
\end{proof}

Nach Umkehrsatz existieren $U \subset \mathbb{R}^k \times S, \ V \subset \mathbb{R}^n$ offen mit
$(p,0) \in U$ und $f(p) = F(p,0) \in V$, sodass $F\vert_U : U \to V$ ein Diffeomorphismus ist.
Nach Definition von $F$ gilt $F(U \cap \mathbb{R}^k \times \{0\}) =  \underbrace{f(\mathbb{R}^k)}_\textrm{$\bild(f)$} \ \cap \ V$. \\
Wir schliessen, dass $\bild (f) \subset \mathbb{R}^n$ lokal um den Punkt $f(p)$ die Bedingungen einer Untermannigfaltigkeit erfüllt. 
\end{proof}

\begin{remark}
Sei $f: \mathbb{R}^k \to \mathbb{R}^n$ eine $C^1$-Abbildung, welche in jedem Punkt eine lokale Einbettung ist.
Dann braucht $f$ nicht injektiv zu sein. Weiterhin ist $f(\mathbb{R}^k) \subset \mathbb{R}^n$ im Allgemeinen keine Untermannigfaltigkeit.
\end{remark}
\begin{example}
Betrachte die Funktion
\begin{align*}
    f \colon \mathbb{R} & \to \mathbb{R}^2 \\
    t & \mapsto (t^2,t^3-t)
\end{align*}
Berechne $(Jf)_t = \begin{pmatrix}2t \\ 3t^2 -1\end{pmatrix} \not  = 0 \ \forall t \in \mathbb{R}$
$\implies (Df)_t : \mathbb{R} \to \mathbb{R}^2$ ist injektiv für alle $t \in \mathbb{R}$
\\ \\
\ctikzfig{counterexample}
Wir sehen, dass f nicht injektiv ist. Also ist $\bild (f)$ keine Untermannigfaltigkeit (lokal um den Punkt $(1,0)$).
\end{example}

\section{Der Tangentialraum einer Untermannigfaltigkeit}

\begin{goal}
Beschreibung der Menge aller Tangentialvektoren in einem Punkt einer Untermannigfaltigkeit.
\end{goal}

\begin{definition}
Sei $M \subset \mathbb{R}^n$ eine $C^1$-UMF der Dimension $k$ und $p \in M$. Wähle eine lokale $C^1$-Parametrisierung $\varphi : U \to M$ um $p$, d.h. $U \subset \mathbb{R}^k$ offen, $\varphi : U \to
\mathbb{R}^n$ stetig differenzierbar mit $\varphi(U) \subset M, \varphi (0)=p \ (0 \in U), \varphi$ injektiv und für alle $q\in U$ ist $(D\varphi)_q : \mathbb{R}^k \to \mathbb{R}^n $ injektiv. Setze
\begin{align*}
    T_pM = \bild\left((D\varphi)_0\right) < \mathbb{R}^n
\end{align*} der \emph{Tangentialraum} von $M$ bei $p$.
\end{definition}

\begin{remark}
Die Dimension von $T_pM < \mathbb{R}^n$ ist $k$, da $(D\varphi)_0 : \mathbb{R}^k \to \mathbb{R}^n$ injektiv ist. Im Falle $k=2$ (d.h. $M$ ist eine Fläche) nennen wir $T_pM$ \emph{Tangentialebene} 
\end{remark}


\begin{example}
Sei $h : \mathbb{R}^k \to \mathbb{R}$ in $C^1$. Betrachte den Graphen $\Gamma = \{(x,y,z) \in \mathbb{R}^3 | \ z = h(x,y)\} \subset \mathbb{R}^3$ und die (globale) Parametrisierung
\begin{align*}
    \varphi : \mathbb{R}^2 & \to \Gamma \subset \mathbb{R}^3 \\
    (u,v) & \mapsto (u,v, h(u,v))
\end{align*}
Bereche die Jakobimatrix im Punkt $q=(u,v)$
\begin{align*}
    &(J\varphi)_{(u,v)}= \begin{pmatrix}1 & 0
        \\0 & 1
        \\ \frac{\partial h}{\partial u}(u,v) & \frac{\partial h}{\partial v}(u,v)\end{pmatrix}
    \text{Rang }2 \implies (D\varphi )_q : \mathbb{R}^2 \to \mathbb{R}^3 \text{ injektiv}\\
    & T_{\varphi(q)}\Gamma = \spann\Bigg\{
        \begin{pmatrix}1 \\ 0 \\ \frac{\partial h}{\partial u}(u,v)\end{pmatrix},
        \begin{pmatrix}0 \\ 1 \\ \frac{\partial h}{\partial v}(u,v)\end{pmatrix}\Bigg\}
\end{align*}
Im Spezialfall $h=0 \colon T_{\varphi(q)}\Gamma = \spann\{e_1, e_2\}$

\end{example}
\begin{lemma}
$T_pM$ hängt nicht von der lokalen Parametrisierung $\varphi : U \to M$ ab.
\end{lemma}
\begin{proof}
Seien $\varphi_1 : U_1 \to M \quad \varphi_2 : U_2 \to M$ lokale $C^1$-Parametrisierungen mit $\varphi_1(0)=\varphi_2(0) = p \in M$. Setze $V = \varphi_1(U_1) \cap \varphi_2(U_2) \subset M$ offen und
$V_1 = \varphi_1^{-1} \subset U_1$ und $V_2 = \varphi_2^{-1} \subset U_2$. \\
Wir erhalten ein kommutatives Diagramm von Abbildungen
\ctikzfig{diagram_tpm}
Nach der Kettenregel für $\varphi_2^{-1} \circ \varphi_1$ im Punkt $0$ gilt:
\begin{align*}
A = (D\varphi_2^{-1} \circ \varphi_1)_0 = (D\varphi_2^{-1})_{\varphi_1(0)} \circ (D\varphi_1)_0 = 
(D\varphi_2)_0^{-1} \circ (D\varphi_1)_0 \implies (D\varphi_1)_0 = (D\varphi_2)_0 \circ A \\
\end{align*} \\[-2\baselineskip] Damit formen wir um:
\begin{align*}
    \implies T_pM = \bild(D\varphi_1)_0 & = \bild((D\varphi_2)_0 \circ A) \\
    &\overset{(1)}{=} (D\varphi_2)_0 \circ A\left(\mathbb{R}^k\right) \\
    &= (D\varphi_2)_0 \left(\mathbb{R}^k\right) \\
    &= \bild(D\varphi_2)_0
\end{align*}
In (1) wird benutzt, dass $A$ invertierbar ist, mit $A^{-1} = (D\varphi_1)_0^{-1} \circ (D\varphi_2)_0$ und damit ist $A\left(\mathbb{R}^k\right) = \mathbb{R}^k$.
\end{proof}

\subsection*{Interpretation des Tangentialraums via Geschwindigkeitsvektoren}

\begin{proposition}
Sei $p \in M$. Der Tangentialraum $T_pM$ besteht aus allen Geschwindigkeitsvektoren der Form
$\gamma ' (0)$ für $C^1$-Wege $\gamma : (-\varepsilon, \varepsilon) \to M$ mit $\gamma (0)=p$.\\
\ctikzfig{tangentplane}
\end{proposition}
\begin{proof}
\begin{enumerate}[(i)]
    \item Sei $\gamma : (-\varepsilon,\varepsilon) \to M$ stetig differenzierbar mit $\gamma(0)=p$. Betrachte eine lokale $C^1$-Parametrisierung $\varphi : U \to M$ mit $\varphi(0)=p$.\\
    Wähle $\delta >0$, sodass $\gamma\big((-\delta,\delta)\big) \subset \varphi (U)$.
    Definiere $\begin{aligned}[t] \bar \gamma : (-\varepsilon, \varepsilon) & \to \mathbb{R}^k \\
    t &\mapsto \varphi^{-1} \circ \gamma(t)
    \end{aligned}$\\
    Dann gilt $\gamma \vert_{(-\delta,\delta)} = \varphi \circ \bar \gamma$ \\
    \begin{align*}
        \implies \gamma'(0) = \frac{d}{dt}\left(\varphi \circ \bar \gamma\right)(0) = (D\varphi)_{\bar \gamma(0)}\left(\bar \gamma'(0)\right) = (D\varphi)_0 \left(\bar \gamma'(0)\right) \in \bild(D\varphi)_0 = T_pM
    \end{align*}
    \item 
    Sei $v \in T_pM = \bild\big((D\varphi)_0\big)$ für eine lokale $C^1$-Parametrisierung
    $\varphi : U \to M$ mit $\varphi(0)=p$.
    Es existiert also $w \in \mathbb{R}^k$ mit $(D\varphi)_0(w) =v$. \\
    Konstruktion eines Weges $\gamma : (-\delta, \delta) \to M$.
    Wähle $\delta > 0$, sodass für alle $t \in (-\delta, \delta)$ gilt:
    $tw \in U$ (geht, da $U$ offen). Definiere nun:
    \begin{align*}
        \gamma : (-\delta,\delta) &\to M \\
        t & \mapsto \varphi(tw)
    \end{align*}Dann gilt: $\gamma (0) = \varphi(0)=p$
    \begin{align*}
        \implies \gamma'(0) = \frac{d}{dt}\left(\varphi(tw)\right)(0) = (D\varphi)_0(w) = v
    \end{align*}
    \ctikzfig{constructionOfPath}
\end{enumerate}
\end{proof}
\newpage

\subsubsection*{Differenzierbare Abbildung zwischen UMF}
Sei $M \subset \mathbb{R}^n$ eine $C^1$-UMF. Eine Abbildung $f : M \to \mathbb{R}^m$ heisst \emph{differenzierbar} im Punkt $p \in M$, falls ein $U \subset \mathbb{R}^n$ offen existiert mit
$p\in U$, sowie $F: U \to \mathbb{R}^m$ differenzierbar mit $F\vert_{U \cap M} = f \vert_{U \cap M}$.
Insbesondere sind Einschränkungen von differenzierbaren Abb. $f:\mathbb{R}^n \to \mathbb{R}^m$ auf UMF $M \subset \mathbb{R}^n$ differenzierbar (in allen Punkten).

\begin{definition}
Seien $M \subset \mathbb{R}^n, N \subset \mathbb{R}^m C^1$-UMF und $f: M \to N$ stetig differenzierbar,
$p \in M$ (d.h. $f : M \to \mathbb{R}^n$ ist stetig differenzierbar mit $f(M) \subset N)$. Definiere
\begin{align*}
    (Df)_p : T_pM & \to T_{f(p)}N \\
    v & \mapsto (DF)_p(v)
\end{align*}
wobei $F : U \to \mathbb{R}^m$ eine beliebige $C^1$-Einschränkung von $f$ um den Punkt p ist, d.h.
$U \subset \mathbb{R}^n$ offen mit $p \in U$ und $F \vert_{U \cap M} = f \vert_{U \cap M}$. Die Abbildung $(Df)_p$ heisst Differential von $f$ an der Stelle $p \in M$.
\end{definition}

\begin{lemma}
\begin{enumerate}[i.)]
    \item Für alle $v \in T_pM$ gilt $(DF)_p(v) \in T_{f(p)}N$
    \item $(DF)_p(v)$ hängt nicht von der Erweiterung $F$ ab.
\end{enumerate}
\end{lemma}
\begin{proof}
i.) Sei $v \in T_pM$ Wähle $\gamma : (-\varepsilon, \varepsilon) \to M$ mit $\gamma (0) =p$ und
$\gamma'(0)=v$, sowie $\bild(\gamma) \subset U$. Betrachte nun den $C^1$-Weg
$\delta = F \circ \gamma : (-\varepsilon, \varepsilon) \to \mathbb{R}^m$.
Es gilt für alle $ t\in (-\varepsilon, \varepsilon) : \delta(t) = F(\gamma(t)) = f(\gamma(t)) \in N$, da $F\vert_M = f \text{ und } \gamma(t) \in M$. \\
Berechne:
\begin{align*}
    \delta(0) = f(\gamma(0)) = f(p) \implies \delta'(0) = \dfrac{d}{dt}(F \circ \gamma)(0) = (DF)_{\gamma(0)}\big(\gamma'(0)\big) = (DF)_p(v)
\end{align*}Nach Proposition gilt $\delta'(0) \in T_{f(p)}N$, also $(DF)_p(v) \in T_{f(p)}N$. \\

ii.)
Seien $F:U \to \mathbb{R}^m$ und $\bar F : \bar U \to \mathbb{R}^m$ zwei Erweiterungen von $f$ (differenzierbar bei p). Für $v \in T_pM$, wähle $\gamma:(-\varepsilon,\varepsilon)\to M$ wie oben, mit $\bild(\gamma) \subset U \cap \bar U$. Definiere wie unter i) zwei Wege $\delta = F \circ \gamma$ und
$\bar \delta = \bar F \circ \gamma$.
Es gilt $\delta '(0)=(DF)_p(v)$ und $\bar \delta '(0) = (D\bar F)_p(v)$.
Beachte: $\delta$ und $\bar \delta$ stimmen überein mit $f \circ \gamma (t) \implies \delta'(0)=\bar \delta'(0)$
\end{proof}

\begin{remark}
Es gilt die \emph{Kettenregel}. Seien $M \subset \mathbb{R}^m,N\subset \mathbb{R}^n, L \subset \mathbb{R}^l \ C^1$-UMF und $f:M\to N , g:N \to L$ in den Punkten $p \in M$ bzw. $f(p)\in N$ differenzierbar.
Dann ist $g \circ f$ im Punkt $p \in M$ differenzierbar, es gilt
\begin{align*}
    \big(D(g \circ f)\big)_p = (Dg)_{f(p)} \circ (Df)_p : T_pM \to T_{g\circ f(p)}L
\end{align*}
\emph{Grund:} Kettenregel gilt für alle Erweiterungen $F, G$.
\end{remark}

\subsubsection{Beispiel einer differenzierbaren Abbildung zwischen UMF}
Sei $\Sigma \subset \mathbb{R}^3$ der Rotationstorus parametrisiert durch $(0 < a < b)$
\begin{align*}
    \varphi : \mathbb{R}^2 & \to \mathbb{R}^3 \\
    (u,v) & \mapsto \begin{pmatrix}
    (b+a \cos u )\cos v \\ (b + a \cos u ) \sin v \\ a \sin u
    \end{pmatrix}
\end{align*} Betrachte die Abbildung 
\begin{align*}
    f : \Sigma & \to S^2 \subset \mathbb{R}^3\\
    q & \mapsto \dfrac{q}{||q||_2}.  
\end{align*}  $f$ lässt sich zu $F : \mathbb{R}^3 \setminus \{0\} \to S^2$
 erweitern. In Koordinaten: $F(x,y,z) = \left( \frac{x}{r}, \frac{y}{r}, \frac{z}{r} \right)$ mit $r = \sqrt{x^2+y^2+z^2}$. $F$ ist differenzierbar (sogar $C^{\infty}$) deshalb auch $f : \Sigma \to S^2$.
 Für $q=(x,y,z) \not = 0$, berechne die Jakobimatrix
 \begin{align*}
     (JF)_{(x,y,z)}=\begin{pmatrix}
     \frac{1}{r} - \frac{x^2}{r^3} & \frac{-xy}{r^3} & \frac{-xz}{r^3} \\
     \frac{-xy}{r^3} & \frac{1}{r}-\frac{y^2}{r^3} & \frac{-yz}{r^3} \\
     \frac{-xz}{r^3} & \frac{-yz}{r^3} & \frac{1}{r}-\frac{z^2}{r^3}
     \end{pmatrix}
 \end{align*}
Betrachte den Punkt $p = \varphi (0,0) = (a+b,0,0) \in \Sigma$.
$(Jf)_p=\left(\begin{smallmatrix} 0 & 0 & 0 \\ 0 & \frac{1}{a+b} & 0 \\ 0 & 0 & \frac{1}{a+b}\end{smallmatrix}\right)$. \\
Bestimme $T_p\Sigma = \spann\{e_2,e_3\}$ und
$T_{f(p)}S^2 = T_{(1,0,0)}S^2 = \spann\{e_2,e_3\}$. Wir erhalten also
$(Df)_p : T_p\Sigma \to T_{f(p)}S^2$.
Es gilt $(Df)_p(e_2)= \dfrac{1}{a+b}e_2$, bzw. $(Df)_p(e_3)= \dfrac{1}{a+b}e_3$ 


\section{Die erste Fundamentalform}

\begin{recall}
Ein Skalarprodukt auf einem reellen Vektorraum $V$ ist eine Abbildung $\langle \cdot , \cdot \rangle : V \times V \to \mathbb{R}$, welche bilinear, symmetrisch und positiv ist.
\begin{itemize}
    \item Positivität: $\langle v, v \rangle \ge 0 \ \text{für alle } v \in V$ und $\langle v, v \rangle = 0 \iff v = 0$.
\end{itemize}
Jedes Skalarprodukt definiert eine positiv definite \emph{quadratische Form} $q$ d.h.
\begin{enumerate}[(i)]
    \item $\forall v \in V, \lambda \in \mathbb{R} : q(\lambda v) = \lambda ^2 q(v)$
    \item $B(v,w) = q(v+w) - q(v) - q(w)$ ist bilinear in $v,w$.
\end{enumerate}
\begin{align*}
    q : V & \to \mathbb{R} \\
    v & \mapsto \langle v,v \rangle 
\end{align*}
\end{recall}

Sei nun $ \Sigma \subset \mathbb{R}^3$ eine reguläre Fläche. Wir erhalten in jedem Punkt $p \in \Sigma$ ein Skalarprodukt
\begin{align*}
    \langle \cdot , \cdot \rangle _p = T_p \Sigma \times T_p \Sigma & \to \mathbb{R} \\
    \langle v, w \rangle & \mapsto \langle v, w \rangle _{\mathbb{R}^3}
\end{align*}
Das Feld von Skalarprodukten $p \mapsto \langle \cdot , \cdot \rangle _p $ heisst \emph{Riemannsche Metrik} auf $ \Sigma$. Die zugehörige quadratische Form $ I_p : T_p \Sigma \to \mathbb{R}$ heisst \emph{erste Fundamentalform} von $\Sigma$ an der Stelle $p$.
\subsubsection*{Beschreibung durch Koeffizienten}

Sei $\langle \ , \ \rangle : \mathbb{R}^2 \times \mathbb{R}^2$ ein Skalarprodukt.
Schreibe $a = a_1e_1 + a_2e_2, b = b_1e_1 + b_2e_2 \in \mathbb{R}^2$.

Berechne \begin{align*}
    \langle a, b \rangle = \begin{pmatrix}
        a_1 & a_2
    \end{pmatrix} \begin{pmatrix}
        \langle e_1, e_1 \rangle & \langle e_1, e_2 \rangle \\
        \langle e_2, e_1 \rangle & \langle e_2, e_2 \rangle
    \end{pmatrix} \begin{pmatrix}
        b_1 \\ b_2
    \end{pmatrix}
\end{align*}

Wähle eine lokale Parametrisierung $\varphi : U \to \Sigma$ mit $p \in \varphi (U)$.
Wir ziehen die Riemannsche Metrik (auf $\Sigma$) wie folgt auf $U$ zurück:

Sei $q \in U$ und seien $a,b \in \mathbb{R}^2$.
Definiere
\begin{align*}
    \langle a,b \rangle _q = \langle (D\varphi)_q(a), (D\varphi)_q(b)\rangle{_{\varphi(q) \leftarrow { \text{ bedeutet }\langle \cdot, \cdot \rangle \text{ ist auf } \mathbb{R}^3}}}
\end{align*}

Sei $p=\varphi (q)$, dann gilt $T_p\Sigma = \bild (D\varphi)_p < \mathbb{R}^3 = \spann \{ (D\varphi)_q(e_1),(D\varphi)_q(e_2)\}$

\begin{notation}
    Schreibe $q= (u,v) \in U \subset \mathbb{R}^2$
    \begin{align*}
        &\varphi_u (q)= \varphi_u(u,v) = (D\varphi)_q(e_1)\in T_{\varphi(q)}\Sigma \\
        &\varphi_v (q)= \varphi_v(u,v) = (D\varphi)_q(e_2)\in T_{\varphi(q)}\Sigma
    \end{align*}``Spalten der Jakobimatrix $(J\varphi)_q$''
\end{notation}

Die \emph{Koeffizienten} der ersten Fundamentalform bezüglich der Parametrisierung $\varphi: U \to \Sigma$ sind
\begin{align*}
    &E(u,v) = \langle e_1, e_1\rangle_{(u,v)} = \langle \varphi_u(u,v),\varphi_u(u,v)\rangle _{\mathbb{R}^3} \\
    &F(u,v) = \langle e_1, e_2\rangle_{(u,v)} = \langle \varphi_u(u,v),\varphi_v(u,v)\rangle _{\mathbb{R}^3} \\
    &G(u,v) = \langle e_2, e_2\rangle_{(u,v)} = \langle \varphi_v(u,v),\varphi_v(u,v)\rangle _{\mathbb{R}^3} \\
\end{align*}

Für alle $a=a_1e_1+a_2e_2, b=b_1e_1+b_2e_2$ gilt nun
\begin{align*}
    \langle a,b \rangle_{u,v} = \begin{pmatrix}
        a_1 & a_2
    \end{pmatrix}\begin{pmatrix}
        E(u,v) & F(u,v) \\
        F(u,v) & G(u,v)
    \end{pmatrix}\begin{pmatrix}
        b_1 \\ b_2
    \end{pmatrix}
\end{align*}

\begin{examples}
    \leavevmode
    \begin{enumerate}
        \item Sei $Z=\{(x,y,z)\in \mathbb{R}^3 \ | \ x^2 + y^2 =1 \} \subset \mathbb{R}^3$.
        Betrachte die lokale Parametrisierung $\varphi : (0,2\pi) \times \mathbb{R} \to Z$, $\varphi (u,v) = (\cos(u), \sin(u), v)$.
        Berechne$$
        (J\varphi)_{(u,v)}=\begin{pmatrix}
            -\sin(u) & 0 \\
            cos(u) & 0 \\
            0 & 1
        \end{pmatrix}$$
        Erste Spalte: $\varphi_u(u,v)$, zweite Spalte $\varphi_v(u,v)$ und damit
        \[
        \left. 
        \begin{matrix} E(u,v)=\langle \varphi_u,\varphi_u\rangle _{\mathbb{R}^3} = 1 \\ 
            F(u,v)=\langle \varphi_u,\varphi_v\rangle _{\mathbb{R}^3} = 0 \\
            G(u,v)=\langle \varphi_v,\varphi_v\rangle _{\mathbb{R}^3} = 1 \end{matrix} 
        \right\} \begin{pmatrix}
            1 & 0 \\ 0 & 1
        \end{pmatrix}
        \]``Der Zylinder $Z$ ist lokal isometrisch (intrinsische Distanz) zur Ebene''

        \item $S^2 = \{(x,y,z) \in \mathbb{R}^3 \ | \ x^2+y^2+z^2 = 1 \} \subset \mathbb{R}^3$, die Einheitskugel.
        Betrachte die lokale Parametrisierung
        \begin{align*}
            \varphi : (0,2\pi) \times (-\frac{\pi}{2}, \frac{\pi}{2}) &\to S^2 \\
            (u,v) &\mapsto (\cos(u)\cos(v), \sin(u)\cos(v), \sin(v))
        \end{align*}
        Berechne
        $$(J\varphi)_{u,v}= \begin{pmatrix}
            -\sin(u)\cos(v) & -\cos(u)\sin(v) \\
            \cos(u)\cos(v) & -\sin(u)\sin(v) \\
            0 & \cos(v)
        \end{pmatrix} = \varphi _u (u,v), \varphi_v(u,v)$$
        und
        \[
        \left. 
        \begin{matrix} E(u,v)=\langle \varphi_u,\varphi_u\rangle _{\mathbb{R}^3} = \cos^2(v) \\ 
            F(u,v)=\langle \varphi_u,\varphi_v\rangle _{\mathbb{R}^3} = 0 \\
            G(u,v)=\langle \varphi_v,\varphi_v\rangle _{\mathbb{R}^3} = 1 \end{matrix} 
        \right\} \begin{pmatrix}
            \cos^2(v) & 0 \\ 0 & 1
        \end{pmatrix}
        \]

        Alternative lokale Parametrisierung
        \begin{align*}
            \varphi : D^2 &\to S^2 \\
            (u,v) &\mapsto (u,v,\sqrt{1-u^2-v^2})
        \end{align*}
        $$(J\varphi)_{u,v}= \begin{pmatrix}
            1 & 0 \\
            0 & 1 \\
            \frac{-u}{\sqrt{1-u^2-v^2}} & \frac{-v}{\sqrt{1-u^2-v^2}}
        \end{pmatrix}$$
        Wir bemerken $F(u,v)= \frac{uv}{1-u^2-v^2} \not = 0$ (ausser falls $uv=0$)
    \end{enumerate}
\end{examples}

\subsubsection*{Weglänge und Flächeninhalt}
\begin{recall}
    Sei $\alpha: [0,1] \to \mathbb{R}^n \ C^1$.
    Die Weglänge von $\alpha$ ist
    $$\mathcal{L} (\alpha) = \int _0 ^1 \sqrt{\underbrace{\langle\dot{\alpha}(t), \dot{\alpha}(t)\rangle _{\mathbb{R}^3}}_{||\dot{\alpha}(t)||_2 \text{ euklidische Norm}}} \ dt$$
\end{recall}

Sei nun $\Sigma \subset \mathbb{R}^3$ eine reguläre Fläche,
$\varphi : U \to \Sigma$ eine lokale Parametrisierung und $\alpha : [0,1] \to \Sigma \ C^1$ mit $\alpha([0,1])\subset \varphi(U)$.
Definiere 
\begin{align*}
    \beta = \varphi ^{-1} \circ \alpha : [0,1] &\to U \subset \mathbb{R}^2 \\
    t &\mapsto \varphi^{-1}(\alpha(t))
\end{align*}

Schreibe $\beta (t) = u(t)e_1 + v(t)e_2$. Es gilt $\alpha(t) = \varphi(\beta(t))=\varphi(u(t),v(t))$.
Berechne $\mathcal{L}(\alpha)= \int _0^1 ||\dot{\alpha}(t)|| \ dt$.
Berechne \begin{align*}
    \dot{\alpha}(t) &= \frac{d}{dt}(\varphi \circ \beta)(t) \\
    &= (D\varphi)_{\beta(t)}(\dot{\beta}(t)) \\
    &= (D\varphi)_{(u(t), v(t))}(\dot{u}(t)e_1 + \dot{v}(t)e_2) \\
    &= \dot{u}(t) \cdot \varphi_u(u(t),v(t))+\dot{v}(t) \cdot \varphi_v(u(t),v(t))
\end{align*} und 
\begin{align*}
    \langle \dot{\alpha}(t), \dot{\alpha}(t) \rangle &= \begin{pmatrix}
        \dot{u}(t) & \dot{v}(t)
    \end{pmatrix}\begin{pmatrix}
        E & F \\ F & G
    \end{pmatrix}\begin{pmatrix}
        \dot{u}(t) \\ \dot{v}(t)
    \end{pmatrix} \\
    &= \left[ \dot{u}(t)^2 \cdot E + 2\dot{u}(t)\dot{v}(t)F + \dot{v}(t)^2\cdot G \right]_{=(*)}
\end{align*}
$\implies \mathcal{L} (\alpha) = \int _0^1 \sqrt{(*)} \ dt$

\begin{example}
    Parametrisiere $\Sigma = S^2$ wie folgt:
    $$\varphi (u,v)= (\cos(u)\cos(v),\sin(u)\cos(v),\sin(v))$$
    Sei 
    \begin{align*}
        \alpha : [0,2\pi] &\to S^2 \\
        t &\mapsto (\cos(t), \sin(t), 0)
    \end{align*}
    und
    \begin{align*}
        \beta : [0, 2\pi] &\to (0,2\pi) \times (-\frac{\pi}{2}, \frac{\pi}{2}) \\
        t &\mapsto (t,0)
    \end{align*}
    Tatsächlich gilt $\alpha (t)=\varphi(t,0) = (\cos(t)\cos(0),\sin(t)\cos(0),\sin(0))$. Daraus folgt
    $\beta(t)= (u(t), v(t))$ mit $u(t)=t$ und $v(t)=0$
    Daraus ergibt sich
    \begin{align*}
        \mathcal{L}(\alpha) &= \int_0^{2\pi} \sqrt{\dot{u}^2E+2\dot{u}\dot{v}F\cdot \dot{v}^2G} \ dt \\
        &= \int_0^{2\pi} \sqrt{E (u(t),v(t))} \ dt \qquad \text{siehe oben } E(u,v) = \cos(v)^2 \\
        &= \int_0^{2\pi} 1 \ dt = 2\pi \qquad \text{hier } v(t)\equiv 0 \implies E\equiv 1
    \end{align*}
\end{example}

\subsubsection*{Flächeninhalt}
Sei $\Sigma \subset \mathbb{R}^3$ eine reguläre Fläche $\varphi : U \to \Sigma$ eine lokale
Fläche $(C^1)$-Parametrisierung, sowie $A\subset \varphi (U) \subset \Sigma$ ein abgeschlossenes Gebiet
mit stückweise stetig differenzierbarem Rand ($\partial A=\cup  _{i=1}^n \bild(\gamma _i)$ mit $\gamma : [0,1] \to \Sigma \ C^1$)

\begin{figure}[htb]
    \centering
    \def\svgwidth{15em}
    \input{figures/figure1.pdf_tex}
\end{figure}

Setze $B = \varphi ^{-1}(A) \subset (U)$. Definiere
$$Area(A)=\int _B |Area(P(\varphi_u,\varphi_v))| \ dudv$$
wobei $P$ das Parallelogram ist, welches von $\varphi_u, \varphi_v \in \mathbb{R}^3$
aufgespannt wird.

\begin{figure}[htb]
    \centering
    \def\svgwidth{\textwidth}
    \input{figures/figure2.pdf_tex}
\end{figure}

\begin{einschub}[Vektorprodukt in $\mathbb{R}^3$]
    Seien $v = \sum \limits_{i=1}^3 v_ie_i, w = \sum \limits_{j=1}^3 v_je_j \in \mathbb{R}^3$ 
    Definiere
    $$v\times w = \begin{pmatrix}
        v_2w_3 -v_3w_2 \\
        v_3w_1 -v_1w_3 \\
        v_1w_2 -v_2w_1
    \end{pmatrix} \in \mathbb{R}^3$$

    \begin{lemma}
        $$||u\times w ||_2 = ||v||_2 \cdot ||w||_2 \cdot \sin(\measuredangle (v,w)) = |Area(P(v,w))|$$(verwende $0 \le \measuredangle (v,w) \le \pi$)
    \end{lemma}
    \begin{proof}
        Berechne:
        \begin{align*}
            &\langle v,w \rangle _{\mathbb{R}^3}^2 + ||v\times w ||^2 \\
            &= (v_1w_1+v_2w_2+v_3w_3)^2+(v_2w_3-v_3w_2)^2+(v_3w_1-v_1w_3)^2+(v_1w_2-v_2w_1)^2 \\
            &= (v_1^2+v_2^2+v_3^2) \cdot (w_1^2+w_2^2+w_3^2)
        \end{align*}
        Also $\langle v,w\rangle ^2 + ||v\times w ||^2 = ||v||^2||w||^2$ \textcolor{red}{(*)}.
        Schlussendlich folgt aus $$\langle v,w\rangle^2=\big(||v||\cdot||w||\cdot \cos \measuredangle (v,w)\big)^2$$
        und $$\cos^2 + \sin^2 = 1$$dass $$||v\times w||^2 = \big(||v||\cdot||w||\sin \measuredangle(v,w) \big)^2$$ 
    \end{proof}
\end{einschub}
\begin{einschub}[Skalarprodukt in $\mathbb{R}^n$]
    Seien $v = \sum \limits_{i=1}^n v_ie_i, w = \sum \limits_{j=1}^n v_je_j \in \mathbb{R}^n$ 
    Definiere
    $$\langle v,w\rangle = v^Tw = \sum \limits_{i=1}^n v_iw_i \in \mathbb{R}$$
    \begin{remark}
        Der Ausdruck $v^Tw$ gilt bezüglich allen orthonormalen Koordinaten.
    \end{remark}
    \begin{enumerate}[i)]
        \item Spezialfall $v=w \implies \langle v,v \rangle = v^Tv = ||v||^2 \qquad$ (Pythagoras)
        \item $v,w$ allgemein: \\
        Sei $A \in O(\mathbb{R}^n)$, d.h. $A$ beschreibt einen orthonormalen Basiswechsel. Es gilt also
        $A^TA = Id$. Deshalb
        $$\langle Av, Aw \rangle = (Av)^TAw = v^TA^TAw=v^Tw = \langle v,w\rangle $$
    \end{enumerate}
    \begin{lemma}
        $\langle v,w \rangle = ||v||\cdot ||w|| \cdot \cos \measuredangle (v,w)$
    \end{lemma}
    \begin{proof}
        \begin{minipage}[t]{0.8\columnwidth}
            Wir können Koordinaten so wählen, dass (nach Bemerkung) $v=v_1e_1$ (mit $v_1>0$) und $w=w_1e_1+w_2e_2$ gilt.
            Dann gilt
            $$\langle v,w \rangle = \underbrace{v_1}_{=||v||}w_1 = ||v|| \cdot ||w|| \cos \measuredangle(v,w)$$
            \end{minipage}
            % \hspace{0.05\linewidth}
        \begin{minipage}[t]{0.15\columnwidth}
            \begin{figure}[H]
                \centering
                \def\svgwidth{\textwidth}
                \input{figures/figure3.pdf_tex}
            \end{figure}
        \end{minipage}

    \end{proof}
    Mit diesen Lemmata erhalten wir
    \begin{align*}
        Area(A) &= \int _B |Area(P(\varphi _u, \varphi _v))|\ dudv \qquad \text { Lemma 3 }\\
        &= \int _B ||\varphi _u \times \varphi _v || \ dudv \\
        &\overset{\textcolor{red}{(*)}}{=} \int _B \sqrt{\langle \varphi _u, \varphi_u \rangle \langle \varphi _v, \varphi _v\rangle- \langle \varphi _u, \varphi_v\rangle ^2} \ dudv
    \end{align*}
\end{einschub}
Wir erhalten folgende Proposition:
\begin{proposition}
    $$Area(A) = \int _B \sqrt{E(u,v)G(u,v)-F(u,v)^2}\ dudv$$
\end{proposition}
\begin{remark}
    Diese Formel gilt bezüglich jeder lokalen Parametrisierung.
\end{remark}

\begin{examples}
    \leavevmode
    \begin{enumerate}
        \item $S^2 \subset \mathbb{R}^3$
        \begin{align*}
            \varphi : (0, 2\pi) \times (-\frac{\pi}{2}, \frac{\pi}{2}) &\to S^2 \\
            (u,v) &\mapsto (\cos(u)\cos(v), \sin(u)\cos(v), \sin(v))
        \end{align*}
        Flächenelement: $\sqrt{EG-F^2}= \sqrt{\cos(v)^2}=|\cos(v)|$ ($-\frac{\pi}{2}<v<\frac{\pi}{2}$).
        Für $B \subset (0, 2\pi) \times (-\frac{\pi}{2}, \frac{\pi}{2})$ und $A=\varphi(B)$ gilt dann.
        $$Area(A)=\int _B \cos(v) \ dudv$$
        Im Grenzfall $B= (0,2\pi) \times (-\frac{\pi}{2},\frac{\pi}{2})$ erhalten wir
        $$Area(S^2)=\int \limits _0 ^{2\pi} \underbrace{\int \limits _{-\frac{\pi}{2}}^{\frac{\pi}{2}} |\cos(v)| \ dv}_{=2}du = 2\pi \cdot 2 = 4\pi$$
    
        \item $T$ Torus (siehe Serie 4) $2\pi a \cdot 2\pi b$.
    \end{enumerate}
\end{examples}

\begin{zusatz}
    Flächenelement $\sqrt{EG-F^2}$. $EG-F^2 = \det \begin{pmatrix}
        E & F \\ F & G
    \end{pmatrix}$
\end{zusatz}

\end{document}